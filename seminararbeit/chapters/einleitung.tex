\section{Einleitung}
Der Klimawandel stellt eine der zentralen wissenschaftlichen und gesellschaftlichen Herausforderungen des 21. Jahrhunderts dar. 
Seit Beginn der industriellen Revolution ist die globale Mitteltemperatur um etwa $\SI{1.1}{\kelvin}$ angestiegen, wobei dieser Anstieg mit einer erhöhten 
atmosphärischen Kohlenstoffdioxid-Konzentration von präindustriellen 280~ppm auf mittlerweile über 420~ppm korreliert. 
Die physikalischen Grundlagen dieses Zusammenhangs wurden bereits im 19. Jahrhundert von Wissenschaftlern wie Joseph Fourier, John Tyndall 
und Svante Arrhenius erkannt, doch ein vollständiges Verständnis der Strahlungsprozesse erfordert ein detailliertes Studium der Wechselwirkungen 
zwischen Strahlung und atmosphärischen Spurengasen \cite{atmosphericCO2}.

Im Zentrum dieser Untersuchung steht die Frage: Welche strahlungsphysikalischen Mechanismen führen dazu, dass \ce{CO2} als Treibhausgas wirkt, 
und wie lässt sich dessen quantitativer Beitrag zum terrestrischen Energiehaushalt bestimmen? Diese Frage ist von fundamentaler Bedeutung für 
das Verständnis des Klimasystems, da \ce{CO2}, trotz seiner geringen atmosphärischen 
Konzentration, eine zentrale Rolle im Strahlungshaushalt der Erde spielt. Während Wasserdampf den größten Einzelbeitrag zum natürlichen 
Treibhauseffekt leistet, bilden die nicht kondensierenden Treibhausgase wie \ce{CO2} die stabilisierende Grundlage dieses Effekts, deren 
Bedeutung im Folgenden genauer analysiert wird.

Zur Beantwortung dieser Fragestellung werden drei zentrale Aspekte systematisch untersucht: die fundamentalen Strahlungsgesetze, 
die quantenmechanische Basis der wellenlängenselektiven Absorption sowie der Einfluss von \ce{CO2} auf die terrestrische Strahlungsbilanz 
mittels vereinfachter Treibhausmodelle.

Der methodische Ansatz basiert auf einer theoretischen Analyse der relevanten physikalischen Zusammenhänge. Die Berechnung des 
\ce{CO2}-Absorptionsspektrums erfolgt unter Verwendung spektroskopischer Daten aus der HITRAN-Datenbank \cite{hitran2020}, die mittels Python \cite{python} 
ausgewertet werden. Die Quantifizierung des Treibhauseffekts erfolgt durch analytische Strahlungsbilanzmodelle.

Strukturell ist diese Untersuchung wie folgt aufgebaut: Kapitel~2 behandelt die strahlungsphysikalischen Grundlagen und das Strahlungsgleichgewicht der Erde. 
Kapitel~3 untersucht die molekularen Mechanismen der Infrarotabsorption durch \ce{CO2}. Kapitel~4 quantifiziert den Einfluss von \ce{CO2} 
auf die globale Mitteltemperatur. Kapitel~5 fasst die Erkenntnisse zusammen und ordnet sie in den klimawissenschaftlichen Kontext ein.