\section{Molekülphysik des CO\texorpdfstring{$_2$}{2}}
Bei Betrachtung der \autoref{fig:planck_earth_sun} stellt sich die fundamentale Frage: 
\textit{Warum absorbiert CO\textsubscript{2} gerade bei diesen spezifischen Wellenlängen?} 
Die Antwort liegt in der quantenmechanischen Wechselwirkung elektromagnetischer Strahlung 
mit dem CO\textsubscript{2}-Molekül, einem Phänomen von erheblicher Komplexität. Im Rahmen 
dieser Arbeit wird der Fokus auf die für den Treibhauseffekt relevanten Absorptionsprozesse 
von CO\textsubscript{2} gelegt, wobei die quantenmechanischen Grundlagen zur besseren 
Verständlichkeit teilweise vereinfacht dargestellt werden, ohne die vollständige 
molekülphysikalische Theorie zu behandeln.

Für praktische Berechnungen atmosphärischer Strahlungstransfers stehen umfassende 
spektroskopische Datenbanken zur Verfügung. Die HITRAN-Datenbank \cite{hitran2020, DBhitran2020} enthält hochaufgelöste 
Absorptionslinien für über 50 atmosphärische Moleküle und bildet den Standard für 
quantitative Analysen. Die folgenden Abschnitte erläutern die physikalischen 
Grundlagen dieser Absorptionsprozesse, ohne die vollständige quantenmechanische 
Behandlung zu vertiefen.


\subsection{Grundlagen der molekularen Absorption}
\label{sec:basics_absorption}
Ein Photon, das auf ein Gasmolekül trifft, kann entweder absorbiert oder gestreut werden. 
Der Streueffekt ist für die vorliegende Betrachtung jedoch von untergeordneter Bedeutung, 
weshalb die Absorption im Fokus der Betrachtung steht. Wird ein Photon absorbiert, führt dies dazu, 
dass das Energieniveau des Moleküls angehoben wird. Umgekehrt kann ein Molekül sein 
Energieniveau senken, indem es ein Photon emittiert. Die Gesamtenergie eines Moleküls
setzt sich dabei aus verschiedenen Beiträgen zusammen: der elektronischen Energie, 
die durch die Verteilung der Elektronen in den Molekülorbitalen bestimmt wird, der 
Rotationsenergie, die aus der Drehbewegung des gesamten Moleküls um seine Trägheitsachsen 
resultiert, sowie der Schwingungsenergie, die durch die periodischen Relativbewegungen der 
Atomkerne gegeneinander entsteht. \cite[S.~286--287, S.~292]{radiativeHeatTransfer}

Für die Wechselwirkung eines Moleküls mit elektromagnetischer Strahlung im Infrarotbereich ist 
ein zeitlich veränderliches Dipolmoment fundamental. Dies bedeutet, dass sich die Ladungsverteilung 
des Moleküls während der Rotation oder Schwingung periodisch ändern muss, wodurch ein oszillierendes 
elektrisches Dipolmoment entsteht. \cite[S.~273]{demtröder2013molekülphysik}

Nach den Grundsätzen der Quantenmechanik können die Energieniveaus für molekulare Elektronenorbitale, 
ebenso wie die Energieniveaus für molekulare Rotation und Vibration, nur bestimmte diskrete 
Werte annehmen. Ein Photon kann nur dann absorbiert werden, wenn seine Energie 
$E = h\nu$ exakt der Energiedifferenz zwischen zwei erlaubten Zuständen entspricht 
(Resonanzbedingung). Daher können nur Photonen mit ganz bestimmten Frequenzen absorbiert 
oder emittiert werden, was zu charakteristischen Spektrallinien führt. \cite[S.~287]{radiativeHeatTransfer}


\subsection{Molekülstruktur und Schwingungsmoden}
\label{sec:vibration_modes}
\autoref{sec:basics_absorption} stellte bereits die drei möglichen Arten der Energieniveaus eines Moleküls dar. Jedoch ist die Energie,
welche nötig ist, um elektronische Übergänge zu bewirken, so groß, dass man elektromagnetische Strahlung mit sehr niedriger
Wellenlänge benötigt (zwischen $\SI{0.01}{\micro\meter}$ und $\SI{1.5}{\micro\meter}$). \cite[S.~287]{radiativeHeatTransfer}
Da sich diese Arbeit auf CO\textsubscript{2} konzentriert und dieses keine relevanten Absorptionsbanden für den hier betrachteten Wellenlängenbereich unter 
$\SI{1.5}{\micro\meter}$ besitzt (siehe \autoref{fig:co2_spektrum_under_1_5}), werden diese Energieänderungen vernachlässigt.
\ifthenelse{\boolean{abbildungen}}{
    \begin{figure}[H]
        \centering
        \includegraphics[width=1\textwidth]{assets/co2_absorption_under_1_5.pdf}
        \caption{Absorptionsspektrum von CO\textsubscript{2} unter $\SI{1.5}{\micro\meter}$. Das einzige erkennbare Absorptionsband bei circa $\SI{1.435}{\micro\meter}$ ist nur knapp über $1.75\%$ Absorption. Daten kommen von der HITRAN Datenbank \autocite{DBhitran2020}, Grafik wurde mithilfe von \autocite{python} erstellt, der Code ist im Anhang \ref{lst:co2_spektrum_small} dokumentiert.}
        \label{fig:co2_spektrum_under_1_5}
    \end{figure}
}{}
Vibrationsübergänge erfordern deutlich weniger Energie, weshalb die zugehörigen Spektrallinien typischerweise zwischen $\SI{1.5}{\micro\meter}$ und 
$\SI{10}{\micro\meter}$ auftreten.Rotationsübergänge benötigen die geringste Energie und erscheinen daher bei Wellenlängen oberhalb von $\SI{10}{\micro\meter}$. 
In der Praxis werden Vibrationsübergänge jedoch fast immer von simultanen Rotationsübergängen begleitet. Daraus resultieren eng benachbarte Spektrallinien,
die teilweise überlappen und die sogenannten \textit{Vibrations-Rotations-Bänder} im Infrarotspektrum formen, die im Fokus dieser Arbeit stehen.
Diese entstehen durch die \textit{Freiheitsgrade} eines Moleküls. Jedes Molekül bewegt sich im dreidimensionalen Raum, weshalb jedes Molekül drei 
translatorische Freiheitsgrade für die x-, y- und z-Richtung besitzt. Da jedes Atom eines Moleküls sich theoretisch im dreidimensionalen Raum bewegen kann, 
beträgt die Gesamtzahl der Freiheitsgrade eines Moleküls mit $N$ Atomen $3N$. Da die Atome im Molekül miteinander verbunden sind, können sie sich nicht 
unabhängig voneinander bewegen, sondern nur relativ zueinander. Diese Bewegungen entsprechen den Vibrations- und Rotations-Freiheitsgraden. Die Anzahl dieser 
inneren Freiheitsgrade beträgt $3N - 3$, also die Gesamtzahl der Freiheitsgrade minus die drei translatorischen Freiheitsgrade. Je nachdem, ob das Molekül linear 
oder nichtlinear ist, werden diese inneren Freiheitsgrade unterschiedlich auf Rotations- und Vibrations-Freiheitsgrade aufgeteilt (siehe \autoref{fig:degrees_of_freedom}). 
Für lineare Moleküle gibt es zwei Rotations-Freiheitsgrade. Für CO\textsubscript{2} ergeben sich daher $3 \cdot 3 - 3 - 2 = 4$ Vibrations-Freiheitsgrade.
 \cite[S.~292--293]{radiativeHeatTransfer}

\ifthenelse{\boolean{abbildungen}}{
    \begin{figure}[H]
        \centering
        \includegraphics[width=1\textwidth]{assets/degrees_of_freedom.pdf}
        \caption{Rotations- und Vibrations-Freiheitsgrade für (a) zweiatomige, (b) linear dreiatomige und (c) nichtlineare dreiatomige Moleküle \cite[S.~293]{radiativeHeatTransfer}. Bild wurde mithilfe von \cite{bigjpg} hochskaliert.}
        \label{fig:degrees_of_freedom}
    \end{figure}
}{}

\subsection{Quantenmechanische Grundlagen der Absorption}
\subsubsection{Rotationsübergänge}
Zur Vereinfachung wird angenommen, dass ein Molekül aus Punktmassen besteht, die durch starre, masselose Stäbe verbunden sind, 
das sogenannte Modell des starren Rotators (rigid rotator model). Mit diesem Modell lassen sich 
die erlaubten Rotationsenergieniveaus mittels der Schrödinger-Gleichung berechnen. Die Lösung dieser Gleichung für lineare Moleküle besagt, 
dass die möglichen Energieniveaus wie folgt gegeben sind \cite[S.~293]{radiativeHeatTransfer}\cite[S.~66]{herzberg}:
\ifthenelse{\boolean{formeln}}{
    \begin{equation}
        E_j = \frac{\hbar^2 }{2I}j(j + 1) = hc_0Bj(j+1), \quad j = 0,1,2,... 
        \quad \text{\cite[S.~293]{radiativeHeatTransfer}\cite[S.~68--71]{herzberg}}
        \label{eq:rigid_rotator}
    \end{equation}
}{}

Dabei ist $I$ das Trägheitsmoment des Moleküls und $\hbar$ die reduzierte Planck-Konstante. Zur Vereinfachung wurde
$B$ eingeführt, das die \textit{Rotationskonstante} beschreibt. Die Rotationsquantenzahl $j$ beschreibt den Rotationszustand
des Moleküls. Für den Grundzustand gilt $j = 0$, bei dem das Molekül nicht rotiert. Je höher $j$ ist, desto schneller ist die 
Rotationsgeschwindigkeit des Moleküls. \cite[S.~294]{radiativeHeatTransfer}

Die für die Absorption bei einem Rotationsübergang erforderliche Energie lässt sich mit $\Delta E = E_{j + 1} - E_j$ berechnen, wodurch
auch die entsprechende Frequenz bestimmt werden kann ($E = h\nu$). Eine notwendige Bedingung für die Beobachtung solcher Übergänge ist jedoch, 
dass das Molekül ein permanentes elektrisches Dipolmoment besitzt. Da CO\textsubscript{2} aufgrund seiner symmetrischen linearen Struktur kein 
permanentes Dipolmoment aufweist, können reine Rotationsübergänge nicht beobachtet werden. \cite[S.~294]{radiativeHeatTransfer} Warum 
Rotationsübergänge trotzdem für CO\textsubscript{2} von Bedeutung sind, wird in \autoref{sec:combined_transitions} erläutert.

\subsubsection{Vibrationsübergänge}
Zur Vereinfachung wird angenommen, dass zwei Punktmassen durch eine perfekt elastische masselose Feder verbunden sind. Dieses Modell
wird als harmonische Oszillator bezeichnet. Bei diesem Modell lassen sich die möglichen Energieniveaus ebenfalls mit einer Lösung der 
Schrödinger-Gleichung berechnen: \cite[S.~294--295]{radiativeHeatTransfer}\cite[S.~73--74]{herzberg}

\ifthenelse{\boolean{formeln}}{
    \begin{equation}
        E_{\upsilon} = h\nu_e\left(\upsilon + \frac{1}{2}\right), \quad \upsilon = 0,1,2,... 
        \quad \text{\cite[S.~259]{radiativeHeatTransfer}\cite[S.~76]{herzberg}}
        \label{eq:harmonic_oscillator}
    \end{equation}
}{}

Dabei ist $\nu_e$ die Eigenfrequenz der harmonischen Schwingung und $\upsilon$ ist die Vibrationsquantenzahl, die
ähnlich wie die Rotationsquantenzahl den Vibrationszustand beschreibt. Dieses Modell lässt nur die Auswahlregel 
$\Delta \upsilon = \pm 1$ zu, sodass die Vibrationszustände nur um $1$ erhöht oder gesenkt werden können. 
Dies würde zu einer einzigen Spektrallinie bei der Eigenfrequenz führen \cite[S.~295]{radiativeHeatTransfer}:
\ifthenelse{\boolean{formeln}}{
    \begin{equation}
        \Delta E = E_{\upsilon + 1} - E_{\upsilon} = h\nu_e\left(\left(\upsilon + \frac{3}{2}\right) - \left(\upsilon + \frac{1}{2}\right)\right) = h\nu_e = \text{const}
    \end{equation}
}{}

\iffalse
$\Delta \upsilon = \pm 1$ folgt daraus das jede quantenmechanische Wellenfunktion der verschiedenen Vibrationszustände eine unterschiedliche Form hat.
Allgemein wechselt es zwischen symetrischen und asymetrischen Wellenformen, der Übergangsmoment (wie stark das Photon mit dem Molekül wechselwirkt)
wird durch ein Integral berechnet das beide Funktionen multipliziert.

Dieses Integral ist nur dann nicht null, wenn die Wellenfunktionen die richtige Symmetrie haben. Das passiert nur bei $\Delta \upsilon = \pm 1$
\fi

Allerdings ist das Modell eines harmonischen Oszillators deutlich ungenauer als das des starren Rotators. Dies lässt sich dadurch erklären,
dass bei einer perfekt elastischen Feder die Kraft linear mit der Auslenkung zunimmt, wohingegen die Kraft bei Atomen, die sich
relativ zueinander bewegen, nichtlinear zunimmt. Wird daher eine komplexere Federkonstante in die Analyse einbezogen, führt dies zu
weiteren Termen in \autoref{eq:harmonic_oscillator}, wodurch sich die Auswahlregel zu $\Delta \upsilon = \pm 1, \pm 2, \pm 3, \ldots$ 
ändert. Dies resultiert in mehreren ungefähr gleichmäßig voneinander entfernten Spektrallinien. Der Übergang mit $\Delta \upsilon = \pm 1$ 
wird als \textit{Fundamentalübergang} bezeichnet und ist normalerweise bei weitem der stärkste, während die weiteren Übergänge mit 
$\Delta \upsilon = \pm 2$, $\Delta \upsilon = \pm 3$, $\ldots$ als Obertöne bezeichnet werden. Der Vibrationszustand eines Moleküls 
wird mithilfe der Vibrationsquantenzahlen beschrieben, bei CO\textsubscript{2} beispielsweise durch $(\upsilon_1, \upsilon_2, \upsilon_3)$. 
Obwohl CO\textsubscript{2} vier Vibrations-Freiheitsgrade besitzt (siehe \autoref{sec:vibration_modes}), 
werden nur drei Quantenzahlen benötigt. Der Grund hierfür ist, dass die zweite Vibrationsmode $\upsilon_2$ 
zweifach entartet ist: Es handelt sich um dieselbe Schwingung, die in zwei senkrecht zueinander 
stehenden Ebenen stattfindet. \cite[S.~295--297]{radiativeHeatTransfer}


\subsubsection{Kombinierte Übergänge}
\label{sec:combined_transitions}
Wären für CO\textsubscript{2} nur Vibrationsübergänge möglich, gäbe es im Absorptionsspektrum nur sehr schmale Spektrallinien.
Dies ist jedoch nicht der Fall (siehe \autoref{fig:co2_spektrum}), da Vibrationsübergänge und Rotationsübergänge oft gleichzeitig stattfinden und zu den
oben genannten \textit{Vibrations-Rotations-Bändern} führen, welche aus vielen nah aneinanderliegenden Spektrallinien bestehen. Für Fundamentalübergänge 
liegt der Mittelpunkt dieser Linien wie beim Modell des harmonischen Oszillators bei der Eigenfrequenz des Vibrationsübergangs $\nu_e$. Dieser Punkt wird
als Bandmitte $\eta_0 = \nu_e/c_0$ bezeichnet. \cite[S.~298]{radiativeHeatTransfer}

Für das einfachste Modell eines starren Rotators (\autoref{eq:rigid_rotator}) kombiniert mit einem harmonischen Oszillator 
(\autoref{eq:harmonic_oscillator}) ergibt sich die Gesamtenergie eines Molekülzustands als Summe der Vibrationsenergie und 
der Rotationsenergie:

\ifthenelse{\boolean{formeln}}{
    \begin{equation}
        E_{\upsilon j} = h\nu_e\left(\upsilon + \frac{1}{2}\right) + hc_0B_{\upsilon }j(j+1), \quad \upsilon, j = 0,1,2,\ldots
        \label{eq:combined_harmonic_rigid}
    \end{equation}
}{}

Im Gegensatz zum idealisierten starren Rotator ist die Rotationskonstante in realen Molekülen 
vom Vibrationszustand abhängig, weshalb hier $B_{\upsilon}$ verwendet wird. Durch die gleichzeitige 
Änderung von Rotations- und Vibrationszustand entstehen bei einem Fundamentalübergang 
($\Delta \upsilon = \pm 1$) drei charakteristische Zweige: der $P$-Zweig ($\Delta j = -1$), 
der $Q$-Zweig ($\Delta j = 0$) und der $R$-Zweig ($\Delta j = +1$). \cite[S.~298]{radiativeHeatTransfer} 
Die Wellenzahlen der Spektrallinien in diesen Zweigen sind gegeben durch:

\begin{subequations}
\begin{align}
    \eta_P(j) &= \eta_0 + j^2(B_{\upsilon+1} - B_{\upsilon}) - j(B_{\upsilon+1} + B_{\upsilon}) \label{eq:etaP}\\
    \eta_Q(j) &= \eta_0 + j^2(B_{\upsilon+1} - B_{\upsilon}) + j(B_{\upsilon+1} - B_{\upsilon}) \label{eq:etaQ}\\
    \eta_R(j) &= \eta_0 + j^2(B_{\upsilon+1} - B_{\upsilon}) + j(3B_{\upsilon+1} - B_{\upsilon}) + 2B_{\upsilon+1} \label{eq:etaR}
\end{align}
\end{subequations}

wobei $\eta_0$ die Bandmitte bezeichnet. Die detaillierte Herleitung dieser Formeln findet sich in \autoref{sec:herleitung_3_zweige}.
Bei linearen Molekülen wie CO\textsubscript{2} gibt es oft keinen $Q$-Zweig. Mehr dazu in \autoref{sec:co2_spektrum}.

\subsection{Das CO\texorpdfstring{$_2$}{2}-Absorptionsspektrum}
\label{sec:co2_spektrum}
\ifthenelse{\boolean{abbildungen}}{
    \begin{figure}[H]
        \centering
        \includegraphics[width=1\textwidth]{assets/co2_absorption.pdf}
        \caption{Absorptionsspektrum von CO\textsubscript{2}. Daten stammen aus der HITRAN-Datenbank \cite{DBhitran2020}, 
        Grafik wurde mithilfe von \autocite{python} erstellt, der Code ist im Anhang \ref{lst:co2_spektrum} dokumentiert.}
        \label{fig:co2_spektrum}
    \end{figure}
}{}
\autoref{fig:co2_spektrum} zeigt die charakteristischen Absorptionsbanden von CO\textsubscript{2}. Die beiden Hauptabsorptionsbanden 
sind die zweifach entartete Biegeschwingung $\upsilon_2$ bei $\approx \SI{667}{\per\centi\meter}$ ($\approx \SI{15.0}{\micro\meter}$) 
und die asymmetrische Streckschwingung $\upsilon_3$ bei $\approx \SI{2349}{\per\centi\meter}$ ($\approx \SI{4.26}{\micro\meter}$). 
Die in \autoref{fig:degrees_of_freedom} dargestellte symmetrische Streckschwingung $\upsilon_1$ bei $\approx \SI{1388}{\per\centi\meter}$ 
($\approx \SI{7.20}{\micro\meter}$) erscheint nicht im Spektrum, da diese Schwingungsmode keine Änderung des Dipolmoments bewirkt und somit 
infrarot-inaktiv ist. \cite[S.~39]{Shimanouchi1972} Das Absorptionsband bei $\approx \SI{2.7}{\micro\meter}$ resultiert aus Kombinationsschwingungen: 
Die Überlagerung von $\upsilon_1 + \upsilon_3$ 
ergibt $\approx \SI{3737}{\per\centi\meter}$ ($\approx \SI{2.68}{\micro\meter}$), während die Kombination $2\upsilon_2 + \upsilon_3$ zu 
$\approx \SI{3683}{\per\centi\meter}$ ($\approx \SI{2.72}{\micro\meter}$) führt. \cite[S.~274]{herzbergPolyatomic} Bei kritischer Beobachtung
fällt auf, dass die symmetrische Streckschwingung $\upsilon_1$ eigentlich infrarot-inaktiv ist. Jedoch wird die Kombination infrarot-aktiv, da 
die gleichzeitige Anregung mit $\upsilon_3$ zu einer Dipolmomentänderung führt.

\subsubsection{Das \texorpdfstring{$\upsilon_3$}{v3}-Vibrationsband von CO\texorpdfstring{$_2$}{2}}
\ifthenelse{\boolean{abbildungen}}{
    \begin{figure}[H]
        \centering
        \includegraphics[width=1\textwidth]{assets/co2_absorption_v3_band.pdf}
        \caption{Absorptionsspektrum des $\upsilon_3$-Vibrationsbandes von CO\textsubscript{2}. Daten kommen von der HITRAN Datenbank \cite{DBhitran2020}
         und von \cite[S.~39]{Shimanouchi1972}, Grafik wurde mithilfe von \cite{python} erstellt, der Code ist im Anhang \ref{lst:co2_spektrum_v3} dokumentiert.}
        \label{fig:co2_spektrum_v3}
    \end{figure}
}{}
Bei genauerer Betrachtung des $\upsilon_3$-Vibrationsbandes in \autoref{fig:co2_spektrum_v3} zeigt sich, dass kein Q-Zweig gebildet wird.
Das lässt sich durch den Drehimpuls eines Photons erklären, jedes Photon trägt einen Drehimpuls von $\hbar$, der bei Absorption auf das
Molekül übertragen werden muss. Deswegen muss es einen kombinierten Rotationsübergang ($\Delta\upsilon = +1$, $\Delta j = \pm 1$) geben. Da unsere Auswahlregel
$\Delta j = 0$ nicht zulässt, ist der Q-Zweig verboten. Die $P$- und $R$-Zweige sind gut sichtbar.

\subsubsection{Das \texorpdfstring{$\upsilon_2$}{v2}-Vibrationsband von CO\texorpdfstring{$_2$}{2}}
\ifthenelse{\boolean{abbildungen}}{
    \begin{figure}[H]
        \centering
        \includegraphics[width=1\textwidth]{assets/co2_absorption_v2_band.pdf}
        \caption{Absorptionsspektrum des $\upsilon_2$-Vibrationsbandes von CO\textsubscript{2}. Daten kommen von der HITRAN Datenbank \autocite{DBhitran2020}
         und von \cite[S.~39]{Shimanouchi1972}, Grafik wurde mithilfe von \autocite{python} erstellt, der Code ist im Anhang \ref{lst:co2_spektrum_v2} dokumentiert.}
        \label{fig:co2_spektrum_v2}
    \end{figure}
}{}
Im Gegensatz zu \autoref{fig:co2_spektrum_v3} zeigt das $\upsilon_2$-Vibrationsband eine deutlich komplexere Struktur. Neben den deutlich
sichtbaren $P$- und $R$-Zweigen sieht man einen stark ausgeprägten $Q$-Zweig. Dieser kommt zustande, da durch die Biegeschwingung $\upsilon_2$
das CO\textsubscript{2} Molekül seine lineare Form verlässt, und da diese Schwingung wie bereits erwähnt zweifach entartet ist, kann
diese Auslenkung in 2 zueinander senkrechten Ebenen stattfinden. Dies ermöglicht das beide Schwingungen z.B. mit $\SI{90}{\degree}$ Phasenverschiebung
schwingen und dadurch einen Drehimpuls um die Molekülachse entsteht (Schwingungsdrehimpuls). Dieses Phänomen ist in \autoref{fig:co2_vibration_illustration} konzeptionell
veranschaulicht. Der Drehimpuls eines absorbierten Photons kann nun entweder die Rotation des Moleküls ändern ($\Delta j = \pm 1$) was zu den
$P$- und $R$-Zweigen führt oder kann in den Schwingungsdrehimpuls fließen ($\Delta j = 0$). \cite[S.~238--240, S~275]{demtröder2013molekülphysik} Die im Vergleich zu den restlichen Zweigen starke
Intensität des $Q$-Zweigs lässt sich durch die \autoref{eq:etaQ} für den $Q$-Zweig erklären. Im Gegensatz zu den anderen Zweigen (\ref{eq:etaP}, \ref{eq:etaR})
ist bei dem Q-Zweig der wichtigste Faktor $B_{\upsilon+1} - B_{\upsilon}$, welcher so klein ist, dass die $Q$-Zweig-Linien unabhängig von $j$ 
nahe bei $\eta_0$ liegen und sich zu einem intensiven Peak überlagern. \cite[S.~90]{banwellFundamentals} Im Gegensatz dazu enthalten die $P$- und $R$-Zweige Terme 
mit $B_{\upsilon+1} + B_{\upsilon}$ wodurch sich die Linien mit zunehmenden $j$ weiter verteilen.
\ifthenelse{\boolean{abbildungen}}{
    \begin{figure}[H]
        \centering
        \includegraphics[width=1\textwidth]{assets/co2_absorption_v2_schwingung.pdf}
        \caption{Konzeptuelle Visualisierung des Schwingungsdrehimpulses des $\upsilon_2$-Vibrationsbandes von CO\textsubscript{2}.
            Grafik wurde mithilfe von \cite{python} erstellt, der Code ist im Anhang \ref{lst:co2_v2_schwingung} dokumentiert.}
        \label{fig:co2_vibration_illustration}
    \end{figure}
}{}

Die quantenmechanische Analyse der CO\textsubscript{2}-Molekülphysik beantwortet die eingangs 
gestellte Frage nach der wellenlängenselektiven Absorption: CO\textsubscript{2} absorbiert 
elektromagnetische Strahlung nur bei jenen Wellenlängen, deren Photonenenergie exakt der 
Energiedifferenz zwischen erlaubten quantenmechanischen Zuständen entspricht. Die spektrale Überlappung der 
CO\textsubscript{2}-Absorptionsbanden mit der terrestrischen Wärmestrahlung 
(siehe \autoref{fig:planck_earth_sun}) erklärt die zentrale Rolle von CO\textsubscript{2} im 
Treibhauseffekt: Während es für die kurzwellige solare Strahlung weitgehend transparent ist, 
absorbiert es die langwellige terrestrische Strahlung effektiv.
