\section{Molekülphysik des CO\texorpdfstring{$_2$}{CO2}}
Die Wechselwirkung elektromagnetischer Strahlung mit atmosphärischen Molekülen 
ist ein quantenmechanisches Phänomen von erheblicher Komplexität. Im Rahmen dieser 
Arbeit liegt der Fokus auf den für den Treibhauseffekt wesentlichen Aspekten, 
ohne die vollständige molekülphysikalische Theorie zu behandeln.

Für praktische Berechnungen atmosphärischer Strahlungstransfers stehen umfassende 
spektroskopische Datenbanken zur Verfügung. Die HITRAN-Datenbank (High-Resolution 
Transmission Molecular Absorption Database) \cite{hitran2020}\cite{DBhitran2020} enthält hochaufgelöste 
Absorptionslinien für über 50 atmosphärische Moleküle und bildet den Standard für 
quantitative Analysen. Die folgenden Abschnitte erläutern die physikalischen 
Grundlagen dieser Absorptionsprozesse, ohne die vollständige quantenmechanische 
Behandlung zu vertiefen.


\subsection{Grundlagen der molekularen Absorption}
Ein Photon, welches auf ein Gasmolekül trifft, kann entweder absorbiert oder gestreut werden. 
Die Streuung ändert die Ausbreitungsrichtung und gegebenenfalls die Energie  des Photons. Dieser Effekt ist für 
unsere Betrachtung jedoch von untergeordneter Bedeutung, weswegen wir uns auf die Absorption konzentrieren. Die 
Absorption eines Photons führt dazu, dass das Energieniveau des Moleküls angehoben wird. Umgekehrt kann ein Molekül 
sein Energieniveau senken, indem es ein Photon emittiert.

Die Gesamtenergie eines Moleküls setzt sich aus verschiedenen Beiträgen zusammen: der elektronischen Energie, 
die durch die Verteilung der Elektronen in den Molekülorbitalen bestimmt wird, der Rotationsenergie, die aus 
der Drehbewegung des gesamten Moleküls um seine Trägheitsachsen resultiert, sowie der Schwingungsenergie, 
die durch die periodischen Relativbewegungen der Atomkerne gegeneinander entsteht.

Die Quantenmechanik besagt, dass die Energieniveaus für molekulare Elektronenorbitale, ebenso wie die 
Energieniveaus für molekulare Rotation und Vibration, nur diskrete Werte annehmen können. Da die Energie 
eines Photons direkt proportional zu seiner Frequenz ist ($E = hf$), müssen Photonen eine bestimmte Frequenz 
haben, um absorbiert oder emittiert zu werden. Dies führt zu diskreten Spektrallinien.

\subsection{Molekülstruktur und Schwingungsmoden}
Wir haben bereits die 3 möglichen Arten der Energieniveaus eines Moleküls kennengelernt. Jedoch ist die Energie
welche nötig ist um den Orbit eines Elektrons zu ändern so groß, das man Elektromagnetische Strahlung mit sehr niedriger
Wellenlänge benötigt (zwischen $\SI{10e-2}{\micro\meter}$ und $\SI{1.5}{\micro\meter}$). \cite[S. ~287]{radiativeHeatTransfer} Da 

\begin{equation}
    E_j = \frac{\hbar^2 }{2I}j(j + 1) = hc_0j(j+1), \quad j = 0,1,2,... \quad \text{\cite[S.~293]{radiativeHeatTransfer}}
\end{equation}

$E = hf = hc_0\eta \Rightarrow \eta = \frac{E}{hc_0}$

\subsection{Molekülstruktur und Schwingungsmoden}

\subsection{Quantenmechanische Grundlagen der Absorption}

\subsection{Das CO\texorpdfstring{$_2$}{CO2}-Absorptionsspektrum}