\section{Molekülphysik des CO\texorpdfstring{$_2$}{CO2}}
Die Wechselwirkung elektromagnetischer Strahlung mit atmosphärischen Molekülen 
ist ein quantenmechanisches Phänomen von erheblicher Komplexität. Im Rahmen dieser 
Arbeit liegt der Fokus auf den durch CO\textsubscript{2} bedingten Aspekten des Treibhauseffekts,
ohne die vollständige molekülphysikalische Theorie zu behandeln.

Für praktische Berechnungen atmosphärischer Strahlungstransfers stehen umfassende 
spektroskopische Datenbanken zur Verfügung. Die HITRAN-Datenbank \cite{hitran2020}\cite{DBhitran2020} enthält hochaufgelöste 
Absorptionslinien für über 50 atmosphärische Moleküle und bildet den Standard für 
quantitative Analysen. Die folgenden Abschnitte erläutern die physikalischen 
Grundlagen dieser Absorptionsprozesse, ohne die vollständige quantenmechanische 
Behandlung zu vertiefen.


\subsection{Grundlagen der molekularen Absorption}
Ein Photon, welches auf ein Gasmolekül trifft, kann entweder absorbiert oder gestreut werden. 
Die Streuung ändert die Ausbreitungsrichtung und gegebenenfalls die Energie  des Photons. Dieser Effekt ist für 
unsere Betrachtung jedoch von untergeordneter Bedeutung, weswegen wir uns auf die Absorption konzentrieren. Die 
Absorption eines Photons führt dazu, dass das Energieniveau des Moleküls angehoben wird. Umgekehrt kann ein Molekül 
sein Energieniveau senken, indem es ein Photon emittiert.

Die Gesamtenergie eines Moleküls setzt sich aus verschiedenen Beiträgen zusammen: der elektronischen Energie, 
die durch die Verteilung der Elektronen in den Molekülorbitalen bestimmt wird, der Rotationsenergie, die aus 
der Drehbewegung des gesamten Moleküls um seine Trägheitsachsen resultiert, sowie der Schwingungsenergie, 
die durch die periodischen Relativbewegungen der Atomkerne gegeneinander entsteht.

Die Quantenmechanik besagt, dass die Energieniveaus für molekulare Elektronenorbitale, ebenso wie die 
Energieniveaus für molekulare Rotation und Vibration, nur diskrete Werte annehmen können. Da die Energie 
eines Photons direkt proportional zu seiner Frequenz ist ($E = h\nu$), müssen Photonen eine bestimmte Frequenz 
haben, um absorbiert oder emittiert zu werden. Dies führt zu diskreten Spektrallinien.


\subsection{Molekülstruktur und Schwingungsmoden}
\label{sec:vibration_modes}
Wir haben bereits die drei möglichen Arten der Energieniveaus eines Moleküls kennengelernt. Jedoch ist die Energie,
welche nötig ist, um elektronische Übergänge zu bewirken, so groß, dass man elektromagnetische Strahlung mit sehr niedriger
Wellenlänge benötigt (zwischen $\SI{0.01}{\micro\meter}$ und $\SI{1.5}{\micro\meter}$)\cite[S.~287]{radiativeHeatTransfer}.
Da wir uns auf CO\textsubscript{2} konzentrieren wollen und dieses keine für uns relevanten Absorptionsbanden im Bereich unter 
$\SI{1.5}{\micro\meter}$ besitzt (siehe \autoref{fig:co2_spektrum_under_1_5}), werden wir diese Energieänderungen vernachlässigen.


\ifthenelse{\boolean{abbildungen}}{
    \begin{figure}[H]
        \centering
        \includegraphics[width=1\textwidth]{assets/co2_absorption_under_1_5.pdf}
        \caption{Absorptionsspektrum von CO\textsubscript{2} unter $\SI{1.5}{\micro\meter}$. Das einzige erkennbare Absorptionsband bei circa $\SI{1.435}{\micro\meter}$ ist nur knapp über $2\%$ Absorption. Daten kommen von der HITRAN Datenbank \autocite{DBhitran2020}, Grafik wurde mithilfe von \autocite{python} erstellt, der Code ist im Anhang \ref{lst:co2_spektrum_small} dokumentiert.}
        \label{fig:co2_spektrum_under_1_5}
    \end{figure}
}{}
Vibrationsübergänge erfordern deutlich weniger Energie, weshalb die zugehörigen Spektrallinien zwischen $\SI{1.5}{\micro\meter}$ und $\SI{10}{\micro\meter}$ auftreten.
Rotationsübergänge benötigen die geringste Energie und erscheinen daher bei Wellenlängen oberhalb von $\SI{10}{\micro\meter}$. In der Praxis werden 
Vibrationsübergänge jedoch fast immer von simultanen Rotationsübergängen begleitet. Daraus resultieren eng benachbarte Spektrallinien,
die teilweise überlappen und die sogenannten \textit{Vibrations-Rotations-Bänder} im Infrarotspektrum formen, auf welche wir uns konzentrieren werden.
Diese entstehen durch die \textit{Freiheitsgrade} eines Moleküls. Jedes Molekül bewegt sich im dreidimensionalen Raum, weswegen jedes Molekül drei 
translatorische Freiheitsgrade für die x-, y- und z-Richtung besitzt. Da jedes Atom eines Moleküls sich theoretisch im dreidimensionalen Raum bewegen kann, 
beträgt die Gesamtzahl der Freiheitsgrade eines Moleküls mit $N$ Atomen $3N$. Da die Atome im Molekül miteinander verbunden sind, können sie sich nicht 
unabhängig voneinander bewegen, sondern nur relativ zueinander. Diese Bewegungen entsprechen den Vibrations- und Rotations-Freiheitsgraden. Die Anzahl dieser 
inneren Freiheitsgrade beträgt $3N - 3$, also die Gesamtzahl der Freiheitsgrade minus die drei translatorischen Freiheitsgrade. Je nachdem, ob das Molekül linear 
oder nicht linear ist, werden diese inneren Freiheitsgrade unterschiedlich auf Rotations- und Vibrations-Freiheitsgrade aufgeteilt. Für lineare Moleküle gibt es 
zwei Rotations-Freiheitsgrade. Für CO\textsubscript{2} ergeben sich daher $3 \cdot 3 - 3 - 2 = 4$ Vibrations-Freiheitsgrade.

\ifthenelse{\boolean{abbildungen}}{
    \begin{figure}[H]
        \centering
        \includegraphics[width=1\textwidth]{assets/degrees_of_freedom.pdf}
        \caption{Rotations- und Vibrations-Freiheitsgrade für (a) zweiatomige, (b) linear dreiatomige und (c) nicht lineare dreiatomige Moleküle. \cite[S.~293]{radiativeHeatTransfer}}
        \label{fig:degrees_of_freedom}
    \end{figure}
}{}

\subsection{Quantenmechanische Grundlagen der Absorption}
\subsubsection{Rotationsübergänge}
Zur Vereinfachung wird angenommen, dass ein Molekül aus Punktmassen besteht, welche durch starre, masselose Stäbe verbunden sind, 
das sogenannte Modell des starren Rotators (rigid rotator model). Mit diesem Modell lassen sich 
die erlaubten Rotationsenergieniveaus mittels der Schrödinger-Gleichung berechnen. Die Lösung dieser Gleichung für lineare Moleküle besagt, 
dass die möglichen Energieniveaus wie folgt gegeben sind \cite[S.~293]{radiativeHeatTransfer}:
\ifthenelse{\boolean{formeln}}{
    \begin{equation}
        E_j = \frac{\hbar^2 }{2I}j(j + 1) = hc_0Bj(j+1), \quad j = 0,1,2,... \quad \text{\cite[S.~293]{radiativeHeatTransfer}}
        \label{eq:rigid_rotator}
    \end{equation}
}{}

Dabei ist $I$ das Trägheitsmoment des Moleküls und $\hbar$ die modifizierte Planck-Konstante. Zur Vereinfachung wurde
$B$ eingeführt, welches die \textit{Rotationskonstante} beschreibt. Die Rotationsquantenzahl $j$ beschreibt den Rotationszustand
des Moleküls. Für den Grundzustand gilt $j = 0$, bei dem das Molekül nicht rotiert. Je höher $j$ ist, desto schneller ist die 
Rotationsgeschwindigkeit des Moleküls \cite[S.~294]{radiativeHeatTransfer}.

Die für die Absorption bei einem Rotationsübergang erforderliche Energie lässt sich mit $\Delta E = E_{j + 1} - E_j$ berechnen, wodurch
auch die entsprechende Frequenz bestimmt werden kann ($E = h\nu$). Eine notwendige Bedingung für die Beobachtung solcher Übergänge ist jedoch, 
dass das Molekül ein permanentes elektrisches Dipolmoment besitzt. Da CO\textsubscript{2} aufgrund seiner symmetrischen linearen Struktur kein 
permanentes Dipolmoment aufweist, können reine Rotationsübergänge nicht beobachtet werden \cite[S.~294]{radiativeHeatTransfer}. Warum 
Rotationsübergänge trotzdem für CO\textsubscript{2} von Bedeutung sind, wird in \autoref{sec:combined_transitions} erläutert.

\subsubsection{Vibrationsübergänge}
Zur Vereinfachung wird angenommen, dass zwei Punktmassen durch eine perfekt elastische masselose Feder verbunden sind. Dieses Modell
wird der harmonische Oszillator genannt. Bei diesem Modell lassen sich die möglichen Energieniveaus ebenfalls mit einer Lösung der 
Schrödinger-Gleichung berechnen \cite[S.~294--259]{radiativeHeatTransfer}:

\ifthenelse{\boolean{formeln}}{
    \begin{equation}
        E_{\upsilon} = h\nu_e\left(\upsilon + \frac{1}{2}\right), \quad \upsilon = 0,1,2,... 
        \quad \text{\cite[S.~259]{radiativeHeatTransfer}}
        \label{eq:harmonic_oscillator}
    \end{equation}
}{}

Dabei ist $\nu_e$ die Eigenfrequenz der harmonischen Schwingung und $\upsilon$ ist die Vibrationsquantenzahl, welche
ähnlich wie die Rotationsquantenzahl den Vibrationszustand beschreibt. Dieses Modell lässt nur die Auswahlregel 
$\Delta \upsilon = \pm 1$ zu, sodass bei diesem Modell die Vibrationszustände nur um $1$ erhöht oder gesenkt werden können. 
Dies würde zu einer einzigen Spektrallinie bei der Frequenz der Eigenfrequenz führen \cite[S.~295]{radiativeHeatTransfer}:
\ifthenelse{\boolean{formeln}}{
    \begin{equation}
        \Delta E = E_{\upsilon + 1} - E_{\upsilon} = h\nu_e\left(\left(\upsilon + \frac{3}{2}\right) - \left(\upsilon + \frac{1}{2}\right)\right) = h\nu_e = \text{const}
    \end{equation}
}{}

\iffalse
$\Delta \upsilon = \pm 1$ folgt daraus das jede quantenmechanische Wellenfunktion der verschiedenen Vibrationszustände eine unterschiedliche Form hat.
Allgemein wechselt es zwischen symetrischen und asymetrischen Wellenformen, der Übergangsmoment (wie stark das Photon mit dem Molekül wechselwirkt)
wird durch ein Integral berechnet das beide Funktionen multipliziert.

Dieses Integral ist nur dann nicht null, wenn die Wellenfunktionen die richtige Symmetrie haben. Das passiert nur bei $\Delta \upsilon = \pm 1$
\fi

Leider ist das Modell eines harmonischen Oszillators deutlich ungenauer als das des starren Rotators. Dies lässt sich dadurch erklären,
dass bei einer perfekt elastischen Feder die Kraft linear mit der Auslenkung zunimmt, wohingegen die Kraft bei Atomen, die sich
relativ zueinander bewegen, nicht linear zunimmt. Wenn man daher eine komplexere Federkonstante in die Analyse einbezieht, führt dies zu
weiteren Termen in \autoref{eq:harmonic_oscillator}, wodurch sich die Auswahlregel zu $\Delta \upsilon = \pm 1, \pm 2, \pm 3, \ldots$ 
ändert. Dies resultiert in mehreren ungefähr gleichmäßig voneinander entfernten Spektrallinien. Der Übergang mit $\Delta \upsilon = \pm 1$ 
wird als \textit{Fundamentalübergang} bezeichnet und ist normalerweise bei weitem der stärkste, während die weiteren Übergänge mit 
$\Delta \upsilon = \pm 2$, $\Delta \upsilon = \pm 3$, $\ldots$ als Obertöne bezeichnet werden. Der Vibrationszustand eines Moleküls 
wird mithilfe der Vibrationsquantenzahlen beschrieben, bei CO\textsubscript{2} beispielsweise durch $(\upsilon_1, \upsilon_2, \upsilon_3)$. 
Obwohl CO\textsubscript{2} vier Vibrations-Freiheitsgrade besitzt (siehe \autoref{sec:vibration_modes}), 
werden nur drei Quantenzahlen benötigt. Der Grund hierfür ist, dass die zweite Vibrationsmode $\upsilon_2$ 
zweifach entartet ist: Es handelt sich um dieselbe Schwingung, die in zwei senkrecht zueinander 
stehenden Ebenen stattfindet \cite[S.~295--297]{radiativeHeatTransfer}.


\subsubsection{Kombinierte Übergänge}
\label{sec:combined_transitions}
Wären für CO\textsubscript{2} nur Vibrationsübergänge möglich, gäbe es im Absorptionsspektrum nur sehr schmale sichtbare Spektrallinien.
Dies ist jedoch nicht der Fall (siehe \autoref{fig:co2_spektrum}), da Vibrationsübergänge und Rotationsübergänge oft gleichzeitig stattfinden und zu den
oben genannten \textit{Vibrations-Rotations-Bändern} führen, welche aus vielen nah aneinanderliegenden Spektrallinien bestehen. Für Fundamentalübergänge 
liegt der Mittelpunkt dieser Linien wie beim Modell des harmonischen Oszillators bei der Eigenfrequenz des Vibrationsübergangs $\nu_e$. Dieser Punkt wird
als Bandmitte $\eta_0 = \nu_e/c_0$ bezeichnet \cite[S.~298]{radiativeHeatTransfer}.

Für das einfachste Modell eines starren Rotators (\autoref{eq:rigid_rotator}) kombiniert mit einem harmonischen Oszillator 
(\autoref{eq:harmonic_oscillator}) ergibt sich die Gesamtenergie eines Molekülzustands als Summe der Vibrationsenergie und 
der Rotationsenergie:

\ifthenelse{\boolean{formeln}}{
    \begin{equation}
        E_{\upsilon j} = h\nu_e\left(\upsilon + \frac{1}{2}\right) + hc_0B_{\upsilon }j(j+1), \quad \upsilon, j = 0,1,2,\ldots
        \label{eq:combined_harmonic_rigid}
    \end{equation}
}{}

Im Gegensatz zum idealisierten starren Rotator ist die Rotationskonstante in realen Molekülen 
vom Vibrationszustand abhängig, weshalb hier $B_{\upsilon}$ verwendet wird. Durch die gleichzeitige 
Änderung von Rotations- und Vibrationszustand entstehen bei einem Fundamentalübergang 
($\Delta \upsilon = \pm 1$) drei charakteristische Zweige: der P-Zweig ($\Delta j = -1$), 
der Q-Zweig ($\Delta j = 0$) und der R-Zweig ($\Delta j = +1$) \cite[S.~298]{radiativeHeatTransfer}. Die Wellenzahlen der 
Spektrallinien in diesen Zweigen sind gegeben durch:

\begin{subequations}
\begin{align}
    \eta_P(j) &= \eta_0 + j^2(B_{\upsilon+1} - B_{\upsilon}) - j(B_{\upsilon+1} + B_{\upsilon})\\
    \eta_Q(j) &= \eta_0 + j(j+1)(B_{\upsilon+1} - B_{\upsilon}) \\
    \eta_R(j) &= \eta_0 + j^2(B_{\upsilon+1} - B_{\upsilon}) + j(3B_{\upsilon+1} - B_{\upsilon}) + 2B_{\upsilon+1}
\end{align}
\end{subequations}

wobei $\eta_0$ die Bandmitte bezeichnet. Die detaillierte Herleitung dieser Formeln findet sich in \autoref{sec:herleitung_3_zweige}. 
Bei der Bandmitte $\eta_0$ selbst tritt keine Spektrallinie auf. Wenn $B_{\upsilon} = B_{\upsilon+1}$ dann gibt es keinen Q-Zweig,
dies ist oft der Fall bei linearen Molekülen wie CO\textsubscript{2}. In \autoref{fig:co2_spektrum_v3} sieht man die $P$- und $R$-Zweige des
$\upsilon_3$ Bandes von CO\textsubscript{2} visualisiert. Der $Q$-Zweig tritt nicht auf.

\ifthenelse{\boolean{abbildungen}}{
    \begin{figure}[H]
        \centering
        \includegraphics[width=1\textwidth]{assets/co2_absorption_v3_band.pdf}
        \caption{Absorptionsspektrum des $\upsilon_3$-Vibrationsbandes von CO\textsubscript{2}. Daten kommen von der HITRAN Datenbank \autocite{DBhitran2020}
         und von \cite[S.~39]{Shimanouchi1972}, Grafik wurde mithilfe von \autocite{python} erstellt, der Code ist im Anhang \ref{lst:co2_spektrum_v3} dokumentiert.}
        \label{fig:co2_spektrum_v3}
    \end{figure}
}{}



\subsection{Das CO\texorpdfstring{$_2$}{CO2}-Absorptionsspektrum}
\ifthenelse{\boolean{abbildungen}}{
    \begin{figure}[H]
        \centering
        \includegraphics[width=1\textwidth]{assets/co2_absorption.pdf}
        \caption{Absorptionsspektrum von CO\textsubscript{2}. Daten kommen von der HITRAN Datenbank \autocite{DBhitran2020}, Grafik wurde mithilfe von \autocite{python} erstellt, der Code ist im Anhang \ref{lst:co2_spektrum_small} dokumentiert.}
        \label{fig:co2_spektrum}
    \end{figure}
}{}