\newpage
\section{Anhang}

\subsection{Literaturverzeichnis}
\printbibliography[heading=none]

\newpage
\pagenumbering{Alph}
\subsection{Daten und Tabellen}

\begin{table}[H]
    \centering
    \begin{tabularx}{0.8\textwidth}{l X}
        \toprule
        \textbf{Art der Oberfläche} & \textbf{Albedo (\%)} \\
        \midrule
        Ozean & 2-10 \\
        Wald & 6-18 \\
        Städte & 14-18 \\
        Grass & 7-25 \\
        Acker & 10-20 \\
        Natürliche Graslandökosysteme & 16-20 \\
        Wüste(Sand) & 35-45 \\
        Eis & 20-70 \\
        Wolken (dünn) & 30 \\
        Wolken (dick) & 60-70 \\
        Schnee (alt) & 40-60 \\
        Schnee (frisch) & 75-95 \\
        \bottomrule
    \end{tabularx}
    \caption{Albedos für Unterschiedliche Oberflächen\cite[S.11]{marshall2007atmosphere}.}
    \label{tab:albedos_surfaces}
\end{table}

\subsection{Quellcode}


\renewcommand{\listoflistingscaption}{{\large Liste aller Quellcodes}}
\listoflistings

\newpage

\begin{code}
\inputminted{python}{../simulationen/formeln/formeln.py}
\caption[Verschiedene implementierungen für physikalische Gesetzte.]{Verschiedene implementierungen für physikalische Gesetzte welche von anderen Quellcodes benutzt werden.}
\label{lst:formeln}
\end{code}

\newpage

\begin{code}
\inputminted{python}{../simulationen/utils.py}
\caption[Verschiedene Hilfs-Funktionen]{Verschiedene Funktionen die in mehreren anderen Quellcodes benötigt werden.}
\label{lst:utils}
\end{code}

\newpage

\begin{code}
\inputminted{python}{../simulationen/wien/__main__.py}
\caption[Planck Funktion für verschiedene Temperaturen mit Wienschem Verschiebungsgesetz.]{Planck Funktion für verschiedene Temperaturen mit Wienschem Verschiebungsgesetz. Benutzt Formeln aus \autoref{lst:formeln}.}
\label{lst:wien}
\end{code}

\newpage

\begin{code}
\inputminted{python}{../simulationen/planck/__main__.py}
\caption[Planck Funktion für Sonne \& Erde mit CO2 Absorptionsbänden.]{Planck Funktion für Sonne \& Erde mit CO2 Absorptionsbänden. Benutzt Formeln aus \autoref{lst:formeln}.}
\label{lst:planck}
\end{code}

\newpage

\begin{code}
\inputminted{python}{../simulationen/co2_spektrum/__main__.py}
\caption[Visualisierung des CO\texorpdfstring{$_2$}{2}-Absorptionsspektrums]{Visualisierung des CO\texorpdfstring{$_2$}{2}-Absorptionsspektrums mithilfe von HITRAN Daten \autocite{DBhitran2020}}
\label{lst:co2_spektrum}
\end{code}


