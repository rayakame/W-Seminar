\newpage
\section{Anhang}

\subsection{Literaturverzeichnis}
\printbibliography[heading=none]

\newpage
\pagenumbering{Alph}
\subsection{Daten und Tabellen}

\begin{table}[H]
    \centering
    \begin{tabularx}{0.8\textwidth}{l X}
        \toprule
        \textbf{Art der Oberfläche} & \textbf{Albedo (\%)} \\
        \midrule
        Ozean & 2-10 \\
        Wald & 6-18 \\
        Städte & 14-18 \\
        Grass & 7-25 \\
        Acker & 10-20 \\
        Natürliche Graslandökosysteme & 16-20 \\
        Wüste(Sand) & 35-45 \\
        Eis & 20-70 \\
        Wolken (dünn) & 30 \\
        Wolken (dick) & 60-70 \\
        Schnee (alt) & 40-60 \\
        Schnee (frisch) & 75-95 \\
        \bottomrule
    \end{tabularx}
    \caption{Albedos für Unterschiedliche Oberflächen\cite[S.11]{marshall2007atmosphere}.}
    \label{tab:albedos_surfaces}
\end{table}

\subsection{Formeln und Herleitungen}
\subsubsection{Herleitung der 3 Übergangsformeln}
\label{sec:herleitung_3_zweige}
Die in \autoref{sec:combined_transitions} beschriebenen 3 Gleichungen für die verschiedenen Zweige eines Vibrations-Rotations-Bands kommen wie folgt zustande:

Als Grundlage wird \autoref{eq:combined_harmonic_rigid} genommen und da wir ja einen Übergang der Energieniveaus brauchen, 
gehen wir erstmal von einem anfänglichen $\upsilon''$, $j''$ und finalen $\upsilon'$, $j'$ Zustand aus:
\ifthenelse{\boolean{formeln}}{
    \begin{align*}
        \Delta E &= E_{j'\upsilon'} - E_{j''\upsilon''} \\
                 &= h\nu_e\left(\upsilon' + \frac{1}{2}\right) + hc_0B_{\upsilon'}j'(j'+1) \\
                 &\quad - \left[h\nu_e\left(\upsilon'' + \frac{1}{2}\right) + hc_0B_{\upsilon''}j''(j''+1)\right] \\
                 &= h\nu_e(\upsilon' - \upsilon'') + hc_0\left[B_{\upsilon'}j'(j'+1) - B_{\upsilon''}j''(j''+1)\right]
    \end{align*}
}{}

Wir betrachten hier nur den Fundamentalübergang einer Absorption, deswegen gilt $\Delta \upsilon = +1$ was zu $\upsilon' = \upsilon'' + 1$ führt. 
Die Notation für beide Zustände wird jetzt zu $\upsilon'' = \upsilon$ und $\upsilon' = \upsilon + 1$ vereinfacht. In die Formel eingesetzt:
\ifthenelse{\boolean{formeln}}{
    \begin{align*}
        \Delta E &= h\nu_e(\upsilon + 1 - \upsilon) + hc_0\left[B_{\upsilon+1}j'(j'+1) - B_{\upsilon}j''(j''+1)\right] \\
                    &= h\nu_e + hc_0\left[B_{\upsilon+1}j'(j'+1) - B_{\upsilon}j''(j''+1)\right]
    \end{align*}
}{}

Umgerechnet in Wellenzahl ergibt sich:
\ifthenelse{\boolean{formeln}}{
    \begin{equation*}
        \Delta\eta = \frac{\Delta E}{hc_0} = \frac{\nu_e}{c_0} + B_{\upsilon+1}j'(j'+1) - B_{\upsilon}j''(j''+1)
    \end{equation*}
}{}

Da $\eta_0 = \nu_e/c_0$:
\ifthenelse{\boolean{formeln}}{
    \begin{equation}
        \Delta\eta = \eta_0 + B_{\upsilon+1}j'(j'+1) - B_{\upsilon}j''(j''+1)
        \label{eq:delta_eta}
    \end{equation}
}{}

Ähnlich wie bei einem vorherigen Schritt führen wir wieder die Vereinfachung $j'' = j$ ein, $j'$ kann auch wieder vereinfacht werden,
je nach Zweig zu $j' = j + 1$, $j' = j$ oder $j' = j - 1$. Für den $P$ Zweig ($j' = j - 1$) ergibt das:
\ifthenelse{\boolean{formeln}}{
        \begin{align*}
j'(j'+1) &= (j-1)(j-1+1) \\
            &= (j-1) \cdot j \\
            &= j^2 - j 
    \end{align*}
}{}

Eingesetzt in \autoref{eq:delta_eta}: 

\ifthenelse{\boolean{formeln}}{
    \begin{align*}
        \eta_P &= \eta_0 + B_{\upsilon+1}(j^2 - j) - B_{\upsilon}j(j+1) \\
                  &= \eta_0 + B_{\upsilon+1}j^2 - B_{\upsilon+1}j - B_{\upsilon}j^2 - B_{\upsilon}j \\
                  &= \eta_0 + j^2(B_{\upsilon+1} - B_{\upsilon}) - j(B_{\upsilon+1} + B_{\upsilon})
    \end{align*}
}{}


\subsection{Quellcode}


\renewcommand{\listoflistingscaption}{{\large Liste aller Quellcodes}}
\listoflistings

\newpage

\begin{code}
\inputminted{python}{../simulationen/formeln/formeln.py}
\caption[Verschiedene implementierungen für physikalische Gesetzte.]{Verschiedene implementierungen für physikalische Gesetzte welche von anderen Quellcodes benutzt werden.}
\label{lst:formeln}
\end{code}

\newpage

\begin{code}
\inputminted{python}{../simulationen/utils.py}
\caption[Verschiedene Hilfs-Funktionen]{Verschiedene Funktionen die in mehreren anderen Quellcodes benötigt werden.}
\label{lst:utils}
\end{code}

\newpage

\begin{code}
\inputminted{python}{../simulationen/wien/__main__.py}
\caption[Planck Funktion für verschiedene Temperaturen mit Wienschem Verschiebungsgesetz.]{Planck Funktion für verschiedene Temperaturen mit Wienschem Verschiebungsgesetz. Benutzt Formeln aus \autoref{lst:formeln}.}
\label{lst:wien}
\end{code}

\newpage

\begin{code}
\inputminted{python}{../simulationen/planck/__main__.py}
\caption[Planck Funktion für Sonne \& Erde mit CO2 Absorptionsbänden.]{Planck Funktion für Sonne \& Erde mit CO2 Absorptionsbänden. Benutzt Formeln aus \autoref{lst:formeln}.}
\label{lst:planck}
\end{code}

\newpage

\begin{code}
\inputminted{python}{../simulationen/co2_spektrum/__main__.py}
\caption[Visualisierung des CO\texorpdfstring{$_2$}{2}-Absorptionsspektrums]{Visualisierung des CO\texorpdfstring{$_2$}{2}-Absorptionsspektrums mithilfe von HITRAN Daten \autocite{DBhitran2020}. Benutzt Funktionen aus \autoref{lst:utils}.}
\label{lst:co2_spektrum}
\end{code}

\newpage

\begin{code}
\inputminted{python}{../simulationen/co2_spektrum_under_1_5/__main__.py}
\caption[Visualisierung des CO\texorpdfstring{$_2$}{2}-Absorptionsspektrums unter 1.5 micrometern]{Visualisierung des CO\texorpdfstring{$_2$}{2}-Absorptionsspektrums unterhalb von $\SI{1.5}{\micro\meter}$ mithilfe von HITRAN Daten \autocite{DBhitran2020}. Benutzt Funktionen aus \autoref{lst:utils}.}
\label{lst:co2_spektrum_small}
\end{code}


