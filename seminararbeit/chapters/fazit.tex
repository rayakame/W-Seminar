\section{Fazit}
Die vorliegende Arbeit untersuchte die strahlungsphysikalischen Mechanismen des \ce{CO2}-Treibhauseffekts 
und dessen quantitativen Beitrag zum terrestrischen Energiehaushalt. Die eingangs gestellte Forschungsfrage 
kann auf Grundlage der theoretischen Analyse und numerischen Berechnungen wie folgt beantwortet werden:

Die Wirkung von \ce{CO2} als Treibhausgas beruht auf einem fundamentalen quantenmechanischen Prinzip: 
Das Molekül absorbiert elektromagnetische Strahlung ausschließlich bei jenen Wellenlängen, deren Photonenenergie 
exakt der Energiedifferenz zwischen erlaubten Vibrations-Rotations-Zuständen entspricht. Die Hauptabsorptionsbanden 
bei $\SI{4.26}{\micro\metre}$ ($\nu_3$-Band) und $\SI{15.0}{\micro\metre}$ ($\nu_2$-Band) liegen dabei im Spektralbereich 
der terrestrischen Wärmestrahlung, während \ce{CO2} für die kurzwellige Sonneneinstrahlung weitgehend transparent bleibt. 
Diese spektrale Selektivität erklärt den asymmetrischen Strahlungstransfer in der Atmosphäre.

Die quantitative Analyse mittels vereinfachter Strahlungsbilanzmodelle ergab, dass die Atmosphäre einen Absorptionsgrad 
von $\varepsilon = \SI{77.8}{\percent}$ aufweist. Der spezifische Beitrag von \ce{CO2} wurde durch Integration über das 
HITRAN-Absorptionsspektrum zu $\varepsilon_{\ce{CO2}} = \SI{23.01}{\percent}$ berechnet, was etwa $\SI{29.6}{\percent}$ 
des gesamten atmosphärischen Absorptionsgrads entspricht. Dieser Wert liegt über den experimentell ermittelten $\SI{20}{\percent}$, 
wobei die Diskrepanz hauptsächlich auf die Modellannahme einer homogenen Atmosphärenschicht zurückzuführen ist. In der 
Realität variieren Parameter wie Druck, Temperatur und Gaskonzentration stark mit der Höhe. Zudem vernachlässigt die 
isolierte Betrachtung von \ce{CO2} Überlappungseffekte mit anderen Treibhausgasen, insbesondere \ce{H2O}.

Trotz dieser Vereinfachungen demonstriert die Arbeit eindeutig, dass \ce{CO2} trotz seiner geringen atmosphärischen 
Konzentration von etwa $\SI{420}{ppm}$ einen messbaren und quantifizierbaren Beitrag zum Treibhauseffekt leistet. 
Die berechnete Temperaturdifferenz von $\SI{33}{\kelvin}$ zwischen der effektiven Planetentemperatur ($\SI{254.6}{\kelvin}$) 
und der gemessenen Oberflächentemperatur ($\SI{288}{\kelvin}$) verdeutlicht die fundamentale Bedeutung des 
atmosphärischen Treibhauseffekts für das Erdklima.

Die Ergebnisse bestätigen die bereits im 19. Jahrhundert von Fourier, Tyndall und Arrhenius erkannten physikalischen Grundlagen: 
\ce{CO2} fungiert als stabilisierendes, nicht kondensierendes Treibhausgas, dessen Konzentration die Basis für 
Rückkopplungsmechanismen wie Wasserdampf bildet.

Für zukünftige Untersuchungen wäre eine Erweiterung auf mehrschichtige Atmosphärenmodelle mit höhenabhängigen Parametern sowie 
die Berücksichtigung von Absorptionsüberlappungen zwischen verschiedenen Treibhausgasen wünschenswert. Dennoch liefert die 
vorliegende Analyse ein fundiertes physikalisches Verständnis der strahlungsphysikalischen Prozesse, die den Zusammenhang 
zwischen atmosphärischer \ce{CO2}-Konzentration und globaler Temperatur bestimmen.