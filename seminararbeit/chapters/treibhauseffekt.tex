\section{Der Treibhauseffekt}
Die Analyse der Strahlungsbilanz in \autoref{sec:strahlungsbilanz_erde} zeigte,
dass die berechnete effektive planetare Temperatur deutlich unter der gemessenen 
globalen Mitteltemperatur liegt. Diese Differenz resultiert aus dem natürlichen
Treibhauseffekt. Der zentrale Mechanismus dieses Effekts und dessen Einfluss auf die Energiebilanz
der Erde werden im folgenden Kapitel untersucht.

\subsection{Mechanismus des Treibhauseffekts}
Der Treibhauseffekt entsteht durch die wellenlängenabhängige Absorption 
elektromagnetischer Strahlung in der Atmosphäre. Während die kurzwellige solare Strahlung 
die Atmosphäre weitgehend ungehindert durchdringt und die Erdoberfläche erwärmt, wird 
die von der Oberfläche emittierte langwellige Infrarotstrahlung teilweise von atmosphärischen 
Spurengasen absorbiert. Die Atmosphäre emittiert ihrerseits thermische Strahlung sowohl 
in Richtung Weltraum als auch zurück zur Erdoberfläche. Diese zusätzliche Gegenstrahlung 
führt zu einer Erhöhung der Oberflächentemperatur gegenüber dem Zustand ohne Atmosphäre.

Die physikalischen Grundlagen dieses Prozesses wurden bereits im 19. Jahrhundert erkannt: 
Joseph Fourier identifizierte 1824 den Mechanismus der atmosphärischen Wärmespeicherung, 
John Tyndall wies dies 1863 experimentell nach, und Svante Arrhenius quantifizierte 1896 erstmals den Zusammenhang 
zwischen CO\textsubscript{2}-Konzentration und globaler Temperatur \cite{atmosphericCO2}.

\subsection{Treibhausmodelle}
Zur quantitativen Beschreibung des Treibhauseffekts dienen vereinfachte Strahlungsbilanzmodelle, 
die den Mechanismus schrittweise veranschaulichen und eine analytische Abschätzung der 
Oberflächentemperatur ermöglichen.

\subsubsection{Das einfache Treibhausmodell}
Das Modell setzt voraus, dass die Atmosphäre vollständig transparent für kurzwellige
Sonnenstrahlung ist, jedoch vollständig opak für Infrarotstrahlung (visualisiert 
in \autoref{fig:easy_greenhouse}). Folglich ist die Gesamtausstrahlung der Erde
von der Temperatur der Atmosphäre $T_a$ abhängig, da nicht die Erdoberfläche direkt, sondern
die Atmosphäre in den Weltraum strahlt. Die Gesamtausstrahlung ergibt sich 
mithilfe des Stefan-Boltzmann-Gesetzes aus \autoref{eq:stefan_boltzmann}:
\ifthenelse{\boolean{formeln}}{
  \begin{equation}
      E_{\text{out}} = \sigma \cdot T_a^4 = A\uparrow 
  \end{equation}
}{}
\ifthenelse{\boolean{abbildungen}}{
    \begin{figure}[H]
        \centering
        \includegraphics[width=0.75\textwidth]{assets/athmosphäre_einfach.pdf}
        \caption{Schematische Darstellung des einfachen Treibhausmodells \cite[S.~15]{marshall2007atmosphere}}
        \label{fig:easy_greenhouse}
    \end{figure}
}{}
Die Energiezufuhr erfolgt durch die Sonneneinstrahlung. Da nur die Querschnittsfläche 
der Erde die Sonnenstrahlung abfängt, muss die Sonneneinstrahlung auf
die gesamte Erdoberfläche umgerechnet werden. Der Faktor für den Anteil der reflektierten
Sonneneinstrahlung (planetare Albedo $\alpha_p$) bleibt gleich (siehe \autoref{sec:strahlungsbilanz_erde}).
\ifthenelse{\boolean{formeln}}{
  \begin{align}
      E_{\text{sun}} &= \frac{\text{interceptierte Sonnenstrahlung}}{\text{Oberfläche der Erde}} = \frac{S_0\pi r^2}{4\pi r^2} = \frac{S_0}{4}\\
      E_{\text{in}} &= (1-\alpha_p)\frac{S_0}{4}
  \end{align}
}{}
In diesem Treibhausmodell existieren zwei fundamentale Energiegleichgewichte:
$E_\text{in} = A\uparrow$, also Gesamteinstrahlung = Gesamtausstrahlung,
und $S\uparrow = E_\text{in} + A\downarrow$, also Ausstrahlung der 
Oberfläche = auf die Oberfläche einfallende Strahlung.

Wird das Stefan-Boltzmann-Gesetz in das erste Energiegleichgewicht eingesetzt,
ergibt sich:
\ifthenelse{\boolean{formeln}}{
  \begin{equation}
      (1-\alpha_p)\frac{S_0}{4} = \sigma \cdot T_a^4
      \label{aljksdjasjdoiwjd}
  \end{equation}
}{}
Für das zweite Energiegleichgewicht folgt:
\ifthenelse{\boolean{formeln}}{
  \begin{equation}
      \sigma \cdot T_s^4 = (1-\alpha_p)\frac{S_0}{4} + \sigma \cdot T_a^4
      \label{oikhaojdhwoadw}
  \end{equation}
}{}
Durch Einsetzen von \autoref{aljksdjasjdoiwjd} in \autoref{oikhaojdhwoadw} ergibt sich:
\ifthenelse{\boolean{formeln}}{
  \begin{align*}
      \sigma \cdot T_s^4 &= 2\sigma \cdot T_a^4 \\
      T_s &= 2^{1/4}T_a
  \end{align*}
}{}

Dies zeigt, dass die Oberflächentemperatur in diesem Modell um den Faktor $2^{1/4} \approx 1{,}19$
größer ist als die effektive planetare Temperatur: $\SI{254.6}{\kelvin} \cdot 1{,}19 \approx \SI{303}{\kelvin}$.
Dieser Wert liegt näher an der globalen Mitteltemperatur der Erdoberfläche von $\SI{288}{\kelvin}$, 
stellt jedoch eine Überschätzung dar. Diese Diskrepanz resultiert aus der 
Modellannahme, dass die Atmosphäre die gesamte Infrarotstrahlung absorbiert. Wie in \autoref{sec:co2_spektrum} 
gezeigt wurde, absorbiert CO\textsubscript{2} jedoch nicht den gesamten Infrarot-Wellenlängenbereich, 
sondern nur bestimmte spektrale Banden. Folglich muss das Treibhausmodell 
dahingehend modifiziert werden, dass es nur einen Teil der Infrarotstrahlung absorbiert.

\subsubsection{Das undichte Treibhausmodell}
Im erweiterten Modell wird angenommen, dass die Atmosphäre einen Teil der von der Oberfläche
emittierten Strahlung absorbiert und einen Teil transmittiert. Das Modell arbeitet mit
drei Temperaturen: $T_a$ bezeichnet die Temperatur der Atmosphäre,
$T_s$ die Oberflächentemperatur und $T_e$ die effektive planetare Temperatur 
(wie in \autoref{sec:strahlungsbilanz_erde} definiert), also die Temperatur der Erde 
als idealer schwarzer Körper ohne Atmosphäre.
\ifthenelse{\boolean{abbildungen}}{
    \begin{figure}[H]
        \centering
        \includegraphics[width=0.75\textwidth]{assets/treibhausmodell_leaky.pdf}
        \caption{Schematische Darstellung des undichten Treibhausmodells \cite[S.~17]{marshall2007atmosphere}}
        \label{fig:leaky_greenhouse}
    \end{figure}
}{}
Das Energiegleichgewicht zwischen Gesamteinstrahlung und Gesamtausstrahlung
lautet in diesem Modell:
\ifthenelse{\boolean{formeln}}{
  \begin{equation}
      E_\text{in} = A\uparrow +  (1 - \varepsilon ) S\uparrow
  \end{equation}
}{}
wobei $\varepsilon$ den Absorptionsgrad der Atmosphäre im Infrarotbereich beschreibt. 
Das Energiegleichgewicht $S\uparrow = E_\text{in} + A\downarrow$ bleibt unverändert. 
Da $A\downarrow = A\uparrow$ gilt, folgt:
\ifthenelse{\boolean{formeln}}{
  \begin{equation*}
      S\uparrow = \frac{2}{2-\varepsilon}E_\text{in}
  \end{equation*}
}{}
Mit $S\uparrow = \sigma T_s^4$ und $E_\text{in} = \sigma T_e^4$ ergibt sich:
\ifthenelse{\boolean{formeln}}{
  \begin{equation}
      T_s = \left( \frac{2}{2-\varepsilon} \right)^{1/4}T_e
  \end{equation}
}{}
Die detaillierte Herleitung befindet sich in \autoref{sec:anhang_treibhausmodell}. Mit dieser
Gleichung lässt sich der reelle Absorptionsgrad der Atmosphäre bestimmen, da für
$T_s = \SI{288}{\kelvin}$ und $T_e = \SI{254.6}{\kelvin}$ 
Werte vorliegen. Aufgelöst nach $\varepsilon$ ergibt sich:
\ifthenelse{\boolean{formeln}}{
  \begin{equation}
    \varepsilon = 2 - \frac{2}{(T_s/T_e)^4} = \SI{77.8}{\percent}
  \end{equation}
}{}
\subsubsection{Berechnung des Absorptionsgrads von CO\texorpdfstring{$_2$}{2}}
Um den spezifischen Beitrag von CO\textsubscript{2} zur Gesamtabsorption zu berechnen, lässt sich
die Absorption für jede Wellenlänge bestimmen und über den gesamten Wellenlängenbereich integrieren.
\ifthenelse{\boolean{formeln}}{
  \begin{equation}
      \varepsilon_{\text{CO}_2}(T) = \frac{\int_0^\infty \varepsilon(\lambda) \cdot E_{b\lambda}(\lambda, T_s) \, d\lambda}{\int_0^\infty E_{b\lambda}(\lambda, T_s) \, d\lambda}
        = \frac{\int_0^\infty \varepsilon(\lambda) \cdot E_{b\lambda}(\lambda, T_s) \, d\lambda}{\sigma T_s^4}
    \end{equation}
}{}
Die wellenlängenabhängige Absorption von CO\textsubscript{2} $\varepsilon(\lambda)$ wurde mithilfe der HITRAN-Datenbank \cite{DBhitran2020} bestimmt.
Die numerische Berechnung erfolgte mittels Python \cite{python}, der zugehörige Code ist in Anhang \ref{lst:co2_absorption} dargestellt. 
Daraus ergibt sich $\varepsilon_{\text{CO}_2} = \SI{22.95}{\percent}$. Nach dieser Berechnung trägt CO\textsubscript{2} mit etwa $\SI{30}{\percent}$ zum
gesamten Absorptionsgrad bei. Experimentelle Daten zeigen jedoch, dass dieser Wert bei etwa $\SI{20}{\percent}$ liegt \cite{atmosphericCO2}. Dieser relative Fehler von ca. $\SI{33}{\percent}$
lässt sich durch verschiedene Faktoren erklären, die aus den Vereinfachungen des verwendeten Treibhausmodells resultieren. Hauptsächlich resultiert die Abweichung daraus,
dass das Modell die Atmosphäre als homogene Schicht behandelt. Tatsächlich variieren Parameter wie Druck, Temperatur und Konzentration
stark mit der Höhe. Zudem stellt die isolierte Betrachtung von CO\textsubscript{2} eine weitere Fehlerquelle dar, da sich das Absorptionsspektrum von CO\textsubscript{2}
beispielsweise mit dem von H\textsubscript{2}O überschneidet.