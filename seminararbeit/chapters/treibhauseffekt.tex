\section{Der Treibhauseffekt}
Die Analyse der Strahlungsbilanz in \autoref{sec:strahlungsbilanz_erde} zeigte,
dass die berechnete effektive planetare Temperatur deutlich unter der gemessenen 
globalen Mitteltemperatur liegt. Diese Differenz resultiert aus dem natürlichen
Treibhauseffekt. Der zentrale Mechanismus dieses Effekts und dessen Einfluss auf die Energiebilanz
der Erde werden im folgenden Kapitel untersucht.

\subsection{Mechanismus des Treibhauseffekts}
Der Treibhauseffekt entsteht durch die wellenlängenabhängige Absorption 
elektromagnetischer Strahlung in der Atmosphäre. Während die kurzwellige solare Strahlung 
die Atmosphäre weitgehend ungehindert durchdringt und die Erdoberfläche erwärmt, wird 
die von der Oberfläche emittierte langwellige Infrarotstrahlung teilweise von atmosphärischen 
Spurengasen absorbiert. Die Atmosphäre emittiert ihrerseits thermische Strahlung sowohl 
in Richtung Weltraum als auch zurück zur Erdoberfläche. Diese zusätzliche Gegenstrahlung 
führt zu einer Erhöhung der Oberflächentemperatur gegenüber dem Zustand ohne Atmosphäre.
 \cite{atmosphericCO2,marshall2007atmosphere}

Die physikalischen Grundlagen dieses Prozesses wurden bereits im 19. Jahrhundert erkannt: 
Joseph Fourier identifizierte 1824 den Mechanismus der atmosphärischen Wärmespeicherung, 
John Tyndall wies dies 1863 experimentell nach, und Svante Arrhenius quantifizierte 1896 erstmals den Zusammenhang 
zwischen \ce{CO2}-Konzentration und globaler Temperatur. \cite{atmosphericCO2}

\subsection{Treibhausmodelle}
Zur quantitativen Beschreibung des Treibhauseffekts dienen vereinfachte Strahlungsbilanzmodelle, 
die den Mechanismus schrittweise veranschaulichen und eine analytische Abschätzung der 
Oberflächentemperatur ermöglichen.

\subsubsection{Das einfache Treibhausmodell}
Das Modell setzt voraus, dass die Atmosphäre vollständig transparent für kurzwellige
Sonnenstrahlung ist, jedoch vollständig opak für Infrarotstrahlung (visualisiert 
in \autoref{fig:easy_greenhouse}). Folglich ist die Gesamtausstrahlung der Erde
von der Temperatur der Atmosphäre $T_a$ abhängig, da nicht die Erdoberfläche direkt, sondern
die Atmosphäre in den Weltraum strahlt. \cite[S.~14--15]{marshall2007atmosphere} Die Gesamtausstrahlung ergibt sich 
mithilfe des Stefan-Boltzmann-Gesetzes aus \autoref{eq:stefan_boltzmann}:
\ifthenelse{\boolean{formeln}}{
  \begin{equation}
      E_{\text{out}} = \sigma \cdot T_a^4 = A\uparrow 
  \end{equation}
}{}
\ifthenelse{\boolean{abbildungen}}{
    \begin{figure}[H]
        \centering
        \includegraphics[width=0.75\textwidth]{assets/athmosphäre_einfach.pdf}
        \caption{Schematische Darstellung des einfachen Treibhausmodells. \cite[S.~15]{marshall2007atmosphere} Bild wurde mithilfe von \cite{bigjpg} hochskaliert.}
        \label{fig:easy_greenhouse}
    \end{figure}
}{}
Die Energiezufuhr erfolgt durch die Sonneneinstrahlung. Da nur die Querschnittsfläche 
der Erde die Sonnenstrahlung abfängt, muss die Sonneneinstrahlung auf
die gesamte Erdoberfläche umgerechnet werden. \cite[S.~14]{marshall2007atmosphere} Der Faktor für den Anteil der reflektierten
Sonneneinstrahlung (planetare Albedo $\alpha_p$) bleibt gleich (siehe \autoref{sec:strahlungsbilanz_erde}).
\ifthenelse{\boolean{formeln}}{
  \begin{align}
      E_{\text{sun}} &= \frac{\text{interceptierte Sonnenstrahlung}}{\text{Oberfläche der Erde}} = \frac{S_0\pi r^2}{4\pi r^2} = \frac{S_0}{4}\\
      E_{\text{in}} &= (1-\alpha_p)\frac{S_0}{4}
  \end{align}
}{}
In diesem Treibhausmodell existieren zwei fundamentale Energiegleichgewichte:
$E_\text{in} = A\uparrow$, also Gesamteinstrahlung = Gesamtausstrahlung,
und $S\uparrow = E_\text{in} + A\downarrow$, also Ausstrahlung der 
Oberfläche = auf die Oberfläche einfallende Strahlung. \cite[S.~15]{marshall2007atmosphere}

Wird das Stefan-Boltzmann-Gesetz in das erste Energiegleichgewicht eingesetzt,
ergibt sich:
\ifthenelse{\boolean{formeln}}{
  \begin{equation}
      (1-\alpha_p)\frac{S_0}{4} = \sigma \cdot T_a^4
      \label{eq:simple_greenhouse_atm_balance}
  \end{equation}
}{}
Für das zweite Energiegleichgewicht folgt:
\ifthenelse{\boolean{formeln}}{
  \begin{equation}
      \sigma \cdot T_s^4 = (1-\alpha_p)\frac{S_0}{4} + \sigma \cdot T_a^4
      \label{eq:simple_greenhouse_surface_balance}
  \end{equation}
}{}
Durch Einsetzen von \autoref{eq:simple_greenhouse_atm_balance} in \autoref{eq:simple_greenhouse_surface_balance} ergibt sich:
\ifthenelse{\boolean{formeln}}{
  \begin{align*}
      \sigma \cdot T_s^4 &= 2\sigma \cdot T_a^4 \\
      T_s &= 2^{1/4}T_a
  \end{align*}
}{}

Dies zeigt, dass die Oberflächentemperatur in diesem Modell um den Faktor $2^{1/4} \approx 1.19$
größer ist als die effektive planetare Temperatur: $\SI{254.6}{\kelvin} \cdot 1{,}19 \approx \SI{303}{\kelvin}$.
Dieser Wert liegt näher an der globalen Mitteltemperatur der Erdoberfläche von $\SI{288}{\kelvin}$, 
stellt jedoch eine Überschätzung dar. Diese Diskrepanz resultiert aus der 
Modellannahme, dass die Atmosphäre sämtliche Infrarotstrahlung absorbiert. Wie in \autoref{sec:co2_spektrum} 
gezeigt wurde, absorbiert \ce{CO2} jedoch nicht den gesamten Infrarot-Wellenlängenbereich, 
sondern nur bestimmte spektrale Banden. Folglich muss das Treibhausmodell 
dahingehend modifiziert werden, dass es nur einen Teil der Infrarotstrahlung absorbiert.

\subsubsection{Das undichte Treibhausmodell}
Im erweiterten Modell wird angenommen, dass die Atmosphäre einen Teil der von der Oberfläche
emittierten Strahlung absorbiert und einen Teil transmittiert. Das Modell arbeitet mit
drei Temperaturen: $T_a$ bezeichnet die Temperatur der Atmosphäre,
$T_s$ die Oberflächentemperatur und $T_e$ die effektive planetare Temperatur 
(wie in \autoref{sec:strahlungsbilanz_erde} definiert), also die Temperatur der Erde 
als idealer schwarzer Körper ohne Atmosphäre. \cite[S.~16]{marshall2007atmosphere}
\ifthenelse{\boolean{abbildungen}}{
    \begin{figure}[H]
        \centering
        \includegraphics[width=0.75\textwidth]{assets/treibhausmodell_leaky.pdf}
        \caption{Schematische Darstellung des undichten Treibhausmodells. \cite[S.~17]{marshall2007atmosphere} Bild wurde mithilfe von \cite{bigjpg} hochskaliert.}
        \label{fig:leaky_greenhouse}
    \end{figure}
}{}
Das Energiegleichgewicht zwischen Gesamteinstrahlung und Gesamtausstrahlung
lautet in diesem Modell:
\ifthenelse{\boolean{formeln}}{
  \begin{equation}
      E_\text{in} = A\uparrow +  (1 - \varepsilon ) S\uparrow
  \end{equation}
}{}
wobei $\varepsilon$ den Absorptionsgrad der Atmosphäre im Infrarotbereich beschreibt. 
Das Energiegleichgewicht $S\uparrow = E_\text{in} + A\downarrow$ bleibt unverändert. \cite[S.~16]{marshall2007atmosphere}
Da $A\downarrow = A\uparrow$ gilt, folgt:
\ifthenelse{\boolean{formeln}}{
  \begin{equation*}
      S\uparrow = \frac{2}{2-\varepsilon}E_\text{in}
  \end{equation*}
}{}
Mit $S\uparrow = \sigma T_s^4$ und $E_\text{in} = \sigma T_e^4$ ergibt sich:
\ifthenelse{\boolean{formeln}}{
  \begin{equation}
      T_s = \left( \frac{2}{2-\varepsilon} \right)^{1/4}T_e
  \end{equation}
}{}
Die detaillierte Herleitung befindet sich in \autoref{sec:anhang_treibhausmodell}. Mit dieser
Gleichung lässt sich der reelle Absorptionsgrad der Atmosphäre bestimmen, da für
$T_s = \SI{288}{\kelvin}$ und $T_e = \SI{254.6}{\kelvin}$ 
Werte vorliegen. Aufgelöst nach $\varepsilon$ ergibt sich:
\ifthenelse{\boolean{formeln}}{
  \begin{equation}
    \varepsilon = 2 - \frac{2}{(T_s/T_e)^4} = \SI{77.8}{\percent}
  \end{equation}
}{}
\subsubsection{Berechnung des Absorptionsgrads von CO\texorpdfstring{$_2$}{2}}
Um den spezifischen Beitrag von \ce{CO2} zur Gesamtabsorption zu berechnen, lässt sich
die Absorption für jede Wellenlänge bestimmen und über den gesamten Wellenlängenbereich integrieren. \cite[S.~728]{bergman2020fundamentals}
\ifthenelse{\boolean{formeln}}{
  \begin{equation}
      \varepsilon_{\text{CO}_2}(T) = \frac{\int_0^\infty \varepsilon(\lambda) \cdot E_{b\lambda}(\lambda, T_s) \, d\lambda}{\int_0^\infty E_{b\lambda}(\lambda, T_s) \, d\lambda}
        = \frac{\int_0^\infty \varepsilon(\lambda) \cdot E_{b\lambda}(\lambda, T_s) \, d\lambda}{\sigma T_s^4}
    \end{equation}
}{}
Die wellenlängenabhängige Absorption von \ce{CO2} $\varepsilon(\lambda)$ wurde mithilfe der HITRAN-Datenbank bestimmt. \cite{DBhitran2020}
Die numerische Berechnung erfolgte mittels Python \cite{python}, der zugehörige Code ist in Anhang \ref{lst:co2_absorption} dargestellt. 
Daraus ergibt sich $\varepsilon_{\text{CO}_2} = \SI{23.01}{\percent}$. Nach dieser Berechnung trägt \ce{CO2} mit etwa $\SI{29.6}{\percent}$ zum
gesamten Absorptionsgrad bei. Experimentelle Daten zeigen jedoch, dass dieser Wert bei etwa $\SI{20}{\percent}$ liegt. \cite{atmosphericCO2} Dieser relative Fehler von ca. $\SI{48}{\percent}$
lässt sich durch verschiedene Faktoren erklären, die aus den Vereinfachungen des verwendeten Treibhausmodells resultieren. Die Abweichung resultiert 
hauptsächlich daraus, dass das Modell die Atmosphäre als homogene Schicht behandelt. Tatsächlich variieren Parameter wie Druck, Temperatur und Konzentration
stark mit der Höhe. Zudem stellt die isolierte Betrachtung von \ce{CO2} eine weitere Fehlerquelle dar, da sich das Absorptionsspektrum von \ce{CO2}
beispielsweise mit dem von H\textsubscript{2}O überschneidet.

\subsection{Spektrale Absorptionseigenschaften der Atmosphäre}
Es wurde bereits erkannt, dass die Annahme, die Atmosphäre absorbiere einen konstanten Anteil $\varepsilon$
der IR-Strahlung, eine starke Vereinfachung darstellt. In Wirklichkeit weist die Atmosphäre eine ausgeprägte spektrale Struktur auf:
\ifthenelse{\boolean{abbildungen}}{
    \begin{figure}[H]
        \centering
        \includegraphics[width=1\textwidth]{assets/atmosphäre_absorption.pdf}
        \caption{Absorptionsspektrum der Atmosphäre. \cite[S.~13]{marshall2007atmosphere} Bild wurde mithilfe von \cite{bigjpg} hochskaliert.}
        \label{fig:atmos_absorption}
    \end{figure}
}{}

Abbildung \ref{fig:atmos_absorption} zeigt, dass die Atmosphäre im sichtbaren Bereich (Maximum der 
Sonneneinstrahlung) nahezu transparent ist, während sie UV-Strahlung fast vollständig absorbiert. 
Im Infrarotbereich variiert die Absorption stark: Bei manchen Wellenlängen ist die Atmosphäre 
transparent (atmosphärisches Fenster), bei anderen nahezu undurchlässig. Bemerkenswert ist, dass 
die Hauptbestandteile N\textsubscript{2} und O\textsubscript{2} praktisch nicht zur Absorption beitragen, O\textsubscript{2} absorbiert 
nur im fernen UV und minimal im nahen IR, während N\textsubscript{2} über den gesamten relevanten Spektralbereich 
transparent bleibt. Der Treibhauseffekt wird somit ausschließlich durch Spurengase wie H\textsubscript{2}O, 
\ce{CO2} und \ce{CH4} verursacht. \cite[S.~14]{marshall2007atmosphere}

\subsection{Strahlungsantrieb durch CO\texorpdfstring{$_2$}{2}-Erhöhung}
Der Strahlungsantrieb $\Delta F$ wird definiert als die Änderung der Netto-Strahlungsenergieflussdichte
(in $\si{\watt\per\meter\squared}$) am Oberrand der Atmosphäre. Ein positiver Strahlungsantrieb
führt zu einer Energiezufuhr zum Klimasystem und damit zu einer Erwärmung, während ein negativer Strahlungsantrieb
eine Abkühlung bewirkt.

Für Änderungen der atmosphärischen \ce{CO2}-Konzentration ergibt sich der Strahlungsantrieb aus detaillierten
Strahlungstransfer-Berechnungen zu \cite[S.~2718]{newRadiativeForcingEstimates}:
\ifthenelse{\boolean{formeln}}{
  \begin{equation}
      \Delta F = 5{,}35 \cdot \ln\left(\frac{C}{C_0}\right) \quad [\si{\watt\per\meter\squared}]
      \label{eq:radiative_forcing_co2}
  \end{equation}
}{}
wobei $C$ die aktuelle \ce{CO2}-Konzentration und $C_0 = \SI{280}{ppm}$ die vorindustrielle Referenzkonzentration
bezeichnet. Die Temperaturantwort des Klimasystems auf einen Strahlungsantrieb wird durch die Klimasensitivität $\partial T_S / \partial Q$
beschrieben. \cite[S.~19]{marshall2007atmosphere}
\ifthenelse{\boolean{formeln}}{
  \begin{equation}
      \Delta T_s = \frac{\partial T_s}{\partial Q} \cdot \Delta F
      \label{eq:climate_sensitivity}
  \end{equation}
}{}

Für den einfachsten Fall ohne Rückkopplungen (nur Schwarzkörper-Rückstrahlung) ergibt 
sich aus der Ableitung des Stefan-Boltzmann-Gesetzes \cite[S.~19]{marshall2007atmosphere}:
\ifthenelse{\boolean{formeln}}{
  \begin{equation}
      \frac{\partial T_s}{\partial Q}\bigg|_{\text{BB}} = \left(4\sigma T_e^3\right)^{-1} \approx \SI{0.26}{\kelvin\per\watt\per\meter\squared}
      \quad \text{\cite[S.~19]{marshall2007atmosphere}}
  \end{equation}
}{}
Für $T_e = \SI{255}{\kelvin}$ folgt für eine \ce{CO2}-Verdopplung ($\Delta F \approx \SI{3.71}{\watt\per\meter\squared}$) 
eine direkte Temperaturerhöhung von nur etwa $\SI{0.99}{\kelvin}$.

Das reale Klimasystem weist jedoch starke Rückkopplungen auf, hauptsächlich durch Wasserdampf. Daher führt im realen Klimasystem eine Verdopplung
von \ce{CO2} zu einer Temperaturerhöhung von $\approx \SI{4}{\kelvin}$. \cite{atmosphericCO2}