\section{Physikalische Grundlagen der Wärmestrahlung}

\subsection{Strahlungsgesetze}
Jedes Medium emittiert elektromagnetische Strahlung zufällig in alle Richtungen. 
Die Intensität dieser Emission hängt sowohl von der Temperatur als auch von den Materialeigenschaften des Mediums ab. 
Der von einer Oberfläche abgegebene Strahlungswärmestrom wird als \textit{spezifische Ausstrahlung} bezeichnet.

Dabei wird zwischen der \textit{gesamten spezifischen Ausstrahlung} $E$ und der \textit{spektralen spezifischen Ausstrahlung} $E_f$ unterschieden:
\begin{align*}
E_f &\equiv \text{abgestrahlte Energie pro Zeit, Oberfläche und Frequenz.} \\
E &\equiv \text{abgestrahlte Energie pro Zeit und Oberfläche.}
\end{align*}
\cite[S.~6--7]{radiativeHeatTransfer}


\subsubsection{Das Plancksche Strahlungsgesetz}

Die spektrale spezifische Ausstrahlung eines ideal schwarzen Körpers $E_b$ wird durch das Plancksche Strahlungsgesetz beschrieben. 
Es gibt an, wie viel Energie pro Zeit, Fläche und Frequenzintervall von einer ideal schwarzen Oberfläche bei einer bestimmten Temperatur $T$ emittiert wird. 
Dieses Gesetz wurde 1900 von Max Planck \cite{plancknormalspektrum} hergeleitet und ist heute als \textit{Plancksches Strahlungsgesetz} bekannt. 
Für eine schwarze Oberfläche, die an ein transparentes Medium mit dem Brechungsindex $n$ grenzt, ergibt sich die spektrale spezifische Ausstrahlung\cite[S.7]{radiativeHeatTransfer} zu:

\begin{equation}
  \label{Plancksche Ausstrahlung Frequenz mit Brechungsindex}
  E_{bf}(T, f) = \frac{2\pi h f^3 n^2}{c_0^2} \cdot \frac{1}{e^{hf/kT}-1} 
  \quad \text{\cite[S.8]{radiativeHeatTransfer}}
\end{equation}

Zur Vereinfachung wird angenommen, dass der Brechungsindex $n = 1$ beträgt, da sich die betrachteten Vorgänge im Vakuum oder in Luft abspielen, wo dieser Wert nahezu identisch ist.

\begin{equation}
  \label{Plancksche Ausstrahlung Frequenz}
  E_{bf}(T, f) = \frac{2\pi h f^3}{c_0^2} \cdot \frac{1}{e^{hf/kT}-1} 
\end{equation}


Das Plancksche Strahlungsgesetz lässt sich auch in Abhängigkeit von der Wellenlänge $\lambda$ formulieren:

\begin{equation}
  \label{Plancksche Ausstrahlung Wellenlänge}
  E_{b\lambda}(T, \lambda) = \frac{2\pi h c_0^2}{\lambda^5} \cdot \frac{1}{e^{hc_0/\lambda kT}-1}
\end{equation}

Dabei bezeichnet $h = \SI{6.626e-34}{\joule\second}$ das Plancksche Wirkungsquantum, 
$c_0 = \SI{2.998e8}{\meter\per\second}$ die Lichtgeschwindigkeit im Vakuum 
und $k = \SI{1.381e-23}{\joule\per\kelvin}$ die Boltzmann-Konstante \cite{codata2018}.

\subsubsection{Das Stefan-Boltzmann Gesetz}
\label{sec:stefan_boltzmann}
Die Integration der spektralen spezifischen Ausstrahlung über das gesamte 
elektromagnetische Spektrum liefert die \textit{Gesamtausstrahlung} $E$:
\begin{equation}
  E = \int_{0}^{\infty}E_f \, df
\end{equation}
\cite{stefanBoltzmannLaw}

Für einen idealen schwarzen Körper setzen wir $E_{bf}$ aus Gleichung 
\eqref{Plancksche Ausstrahlung Frequenz} in das Integral ein:
\begin{equation*}
  E_b(T) = \int_{0}^{\infty} \frac{2\pi h f^3}{c_0^2} \cdot \frac{1}{e^{hf/kT}-1} \, df
\end{equation*}

Die Auswertung dieses Integrals erfordert komplexe Integrationstechniken und ist 
in Integraltabellen dokumentiert\cite[S.13]{radiativeHeatTransfer}. 
Das Ergebnis ist das \textit{Stefan-Boltzmann Gesetz}:
\begin{equation}
    \label{eq:stefan_boltzmann}
  E_b(T) = \frac{2\pi^5k^4}{15c_0^2h^3}T^4 = \sigma T^4 \quad \text{\cite{stefanBoltzmannLaw}}
\end{equation}

Dabei bezeichnet $\sigma = \SI{5.670e-8}{\watt\per\meter\squared\per\kelvin\tothe{4}}$ 
die Stefan-Boltzmann Konstante \cite{codata2018}.

\iffalse
\subsubsection{Wiensches Verschiebungsgesetz}
Die Wellenlänge $\lambda_{max}$ bei welcher die spektrale spezifische Ausstrahlung eines schwarzen idealen Körpers $E_{b\lambda}$ 
mit der Temperatur $T$ ein Maximum erreicht, erhält man indem man die Gleichung \eqref{Plancksche Ausstrahlung Wellenlänge} nach 
$\lambda$ ableitet und diese Gleichung gleich $0$ setzt.

\begin{equation}
  \frac{\partial E_{b\lambda}(T, \lambda)}{\partial \lambda} = 0
\end{equation}

Ausgehend von Gleichung \eqref{Plancksche Ausstrahlung Wellenlänge} und unter Verwendung der Konstanten $C_1$ und $C_2$ 
lautet die Bedingung für das Maximum:

\begin{equation}
  \frac{\partial}{\partial \lambda}\left[\frac{C_1}{n^2\lambda^5[e^{C_2/\lambda T}-1]}\right] = 0
\end{equation}

Durch Anwendung der Produkt- und Kettenregel erhält man:

\begin{equation}
  -\frac{5}{\lambda^6[e^{C_2/\lambda T}-1]} + \frac{C_2 e^{C_2/\lambda T}}{\lambda^7 T[e^{C_2/\lambda T}-1]^2} = 0
\end{equation}

Multiplikation mit $\lambda^7 T [e^{C_2/\lambda T}-1]^2$ und Umformung führt zu:

\begin{equation}
  C_2 e^{C_2/\lambda T} = 5\lambda T[e^{C_2/\lambda T}-1]
\end{equation}

Mit der Substitution $x = \frac{C_2}{\lambda T}$ erhält man die transzendente Gleichung:

\begin{equation}
  x = 5(1 - e^{-x})
\end{equation}

Die numerische Lösung dieser Gleichung liefert $x \approx 4.965$. Daraus ergibt sich das \textit{Wiensche Verschiebungsgesetz}:

\begin{equation}
  \lambda_{max} T = \frac{C_2}{4.965} \approx \SI{2898}{\micro\meter\kelvin}
\end{equation}

oder für ein Medium mit Brechungsindex $n$:

\begin{equation}
  n\lambda_{max} T \approx \SI{2898}{\micro\meter\kelvin}
\end{equation}

Dieses Gesetz beschreibt die Verschiebung des Emissionsmaximums zu kürzeren Wellenlängen bei steigender Temperatur 
und erklärt beispielsweise die Farbänderung eines erhitzten Körpers von rot über gelb zu weiß.



\iffalse

Üblicherweise werden folgende Abkürzungen bei dem Plancksche'n Verschiebungsgesetz \eqref{Plancksche Ausstrahlung Wellenlänge} eingeführt:
\begin{align*}
  C_1 &= 2\pi h c_0^2 = \SI{3.7418}{\watt\meter\squared} \\
  C_2 &= \frac{hc_0}{k} = \SI{14.388}{\micro\meter\kelvin} = \SI{1.4388}{\centi\meter\kelvin}
\end{align*}

so dass die Formel \eqref{Plancksche Ausstrahlung Wellenlänge} umgestellt werden kann zu:

\begin{equation}
  \frac{E_{b\lambda}}{n^3T^5} = \frac{C_1}{(n\lambda T)^5[e^{C_2/(n\lambda T)}-1]} \quad (n = const)
\end{equation}
\fi

\fi

\subsubsection{Wiensches Verschiebungsgesetz}
Die Wellenlänge $\lambda_{max}$ bei welcher die spektrale spezifische Ausstrahlung eines schwarzen idealen Körpers $E_{b\lambda}$ 
mit der Temperatur $T$ ein Maximum erreicht, erhält man indem man die Gleichung \eqref{Plancksche Ausstrahlung Wellenlänge} nach 
$\lambda$ ableitet und diese Gleichung gleich Null setzt. Die folgende mathematische Herleitung folgt Kraus \cite[S.101]{kraus2007atmosphäre}.

\begin{equation*}
  \frac{\partial E_{b\lambda}(T, \lambda)}{\partial \lambda} = 0
\end{equation*}

Die Ableitung nach $\lambda$ ergibt mit der Produktregel:
\begin{equation*}
  -10 \frac{hc_0^2}{\lambda^6} \cdot \frac{1}{e^{hc_0/k\lambda T} - 1} 
- 2 \frac{hc_0^2}{\lambda^5} \cdot \frac{1}{\left(e^{hc_0/k\lambda T} - 1\right)^2} \cdot e^{hc_0/k\lambda T} \cdot \left(-\frac{hc_0}{k\lambda^2 T}\right) = 0
\end{equation*}
\begin{equation*}
  \frac{5}{\lambda} = \frac{1}{e^{hc_0/k\lambda T}-1} \cdot e^{hc_0/k\lambda T} \cdot \frac{hc_0}{k\lambda^2T}
\end{equation*}

Mit der Substitution $x = \frac{hc_0}{k\lambda T}$ ergibt sich die transzendente Gleichung:
\begin{equation*}
  5 = \frac{x\cdot e^x}{e^x -1}
\end{equation*}

Die numerische Lösung dieser Gleichung liefert $x \approx 4.9651$. 
Rücksubstitution in $x_{max} = \frac{hc_0}{k\lambda_{max} T}$ ergibt das \textit{Wiensche Verschiebungsgesetz}:

\begin{align*}
\lambda_{max} &= \frac{h c_0}{k x_{max} T} \\
\lambda_{max} T &= \frac{h c_0}{k x_{max}} \\
&= \frac{\SI{6.62612e-34}{\joule\second} \cdot \SI{2.99792e8}{\meter\per\second}}{\SI{1.38065e-23}{\joule\per\kelvin} \cdot 4.9651}\\
&= \SI{2.8978e-3}{\meter\kelvin}
\end{align*}


Das Wiensche Verschiebungsgesetz besagt, dass sich das Maximum der spektralen Ausstrahlung mit steigender Temperatur zu kürzeren Wellenlängen verschiebt\cite[S.101]{kraus2007atmosphäre}.
Die Konstante $b = \SI{2.8978e-3}{\meter\kelvin}$ wird als \textit{Wiensche Verschiebungskonstante} bezeichnet\cite[S.49]{codata2018}.
Umgestellt nach der Wellenlänge des Maximums ergibt sich:
\begin{equation}
  \label{eq:wiens_displacement_law}
  \lambda_{max} = \frac{b}{T}
\end{equation}

\begin{figure}[H]
    \centering
    \includegraphics[width=0.8\textwidth]{assets/wien_plot.pdf}
    \caption{Spektrale spezifische Ausstrahlung $E_{b\lambda}$ eines schwarzen Körpers nach dem Planckschen Strahlungsgesetz für verschiedene Temperaturen. Die gestrichelte Linie verbindet die Maxima der Planck-Kurven und verdeutlicht das Wiensche Verschiebungsgesetz.}
    \label{fig:planck_wien}
\end{figure}

\subsection{Strahlungsbilanz der Erde}
Die Erde bezieht nahezu ihre gesamte Energie von der Sonne. Um ein thermisches Gleichgewicht aufrechtzuerhalten, muss sie 
Energie mit derselben Rate wieder abstrahlen, mit der sie diese empfängt\cite[S.9]{marshall2007atmosphere}:

\begin{equation}
  P_{\text{in}} = P_{\text{out}}
  \label{eq:power_balance}
\end{equation}

Die einfallende solare Bestrahlungsstärke am oberen Rand der Atmosphäre, die sogenannte \textit{Solarkonstante}, beträgt $S_0 = \SI{1361}{\watt\per\meter\squared}$\cite[S.~5--6]{solarIrradiance}.
Da die Querschnittsfläche der Erde, welche die solare Strahlung abfängt, $\pi r^2$ beträgt\cite[S.11]{marshall2007atmosphere}, wobei $r = \SI{6.371e6}{\meter}$\cite[S.56]{formelsammlung} der Erdradius ist, ergibt sich für die einfallende Strahlungsleistung:
\begin{align}
  P_{\text{solar}} &= S_0\pi r^2 \nonumber\\
         &= \SI{1.735e17}{\watt}
  \label{eq:solar_power}
\end{align}

Jedoch wird nicht die gesamte Strahlung von der Erde absorbiert, da ein Teil reflektiert wird. Der Anteil der reflektierten Strahlung wird als \textit{Albedo} $\alpha$ bezeichnet. 
Abbildung \ref{fig:albedo_map} zeigt, dass $\alpha$ von der reflektierenden Oberfläche abhängt (detaillierte Werte für verschiedene Oberflächen siehe Anhang \autoref{tab:albedos_surfaces}).
Im globalen Mittel wird ein Anteil von $\alpha_p \simeq 0{,}30$ der eingehenden Strahlung reflektiert. Diese Größe wird als \textit{planetare Albedo} bezeichnet\cite[S.11]{marshall2007atmosphere}.\begin{figure}[H]
    \centering
    \includegraphics[width=0.8\textwidth]{assets/albedo_map.pdf}
    \caption{Räumliche Verteilung der Albedo an der Erdoberfläche\cite[S.12]{marshall2007atmosphere}.}
    \label{fig:albedo_map}
\end{figure}

Daraus folgt für die von der Erde absorbierte Strahlungsleistung:
\begin{align}
  P_{\text{in}} &= (1-\alpha_p)S_0\pi r^2 \nonumber\\
         &= \SI{1.215e17}{\watt}
  \label{eq:absorbed_power}
\end{align}

Wegen des Energiegleichgewichts muss die von der Erde emittierte Strahlung die absorbierte Strahlung kompensieren. Dazu kann man annehmen, dass sich die Erde
wie ein idealer schwarzer Körper mit gleichmäßiger Temperatur $T_e$ verhält (bekannt als \textit{effektive planetare Temperatur}) und dass das \textit{Stefan-Boltzmann-Gesetz} wie in \autoref{sec:stefan_boltzmann} anwendbar ist.

\autoref{eq:stefan_boltzmann} gibt die Gesamtausstrahlung $E_b(T)$ an, welche die ausgestrahlte Leistung pro Einheit der Oberfläche beschreibt: $E_b = \frac{P}{A}$\cite{stefanBoltzmannLaw}. 
Die Erde strahlt über ihre gesamte Oberfläche $A = 4\pi r^2$ ab. Umgestellt nach der Strahlungsleistung ergibt sich\cite[S.11]{marshall2007atmosphere}:
\begin{equation}
    P_{\text{out}} = \sigma \cdot 4\pi r^2 \cdot T_e^4
    \label{eq:earth_emission}
\end{equation}

Durch Einsetzen von \autoref{eq:absorbed_power} und \autoref{eq:earth_emission} in \autoref{eq:power_balance} und Umstellen nach $T_e$ erhält man:
\begin{equation}
    T_e = \left(\frac{S_0(1-\alpha_p)}{4\sigma}\right)^{1/4}
    \label{eq:effective_temperature}
\end{equation}

Mit den Werten $S_0 = \SI{1361}{\watt\per\meter\squared}$, $\alpha_p = 0{,}30$ und $\sigma = \SI{5.670e-8}{\watt\per\meter\squared\per\kelvin\tothe{4}}$ ergibt sich:
\begin{align}
    T_e &= \left(\frac{\SI{1361}{\watt\per\meter\squared} \cdot (1-0{,}30)}{4 \cdot \SI{5.670e-8}{\watt\per\meter\squared\per\kelvin\tothe{4}}}\right)^{1/4} \nonumber\\
    &= \SI{254.6}{\kelvin}
    \label{eq:effective_temperature_value}
\end{align}

Diese Temperatur von etwa \SI{-18}{\celsius} liegt deutlich unter der gemessenen globalen Mitteltemperatur der Erdoberfläche von circa \SI{288}{\kelvin}\cite[S.11]{marshall2007atmosphere}. 
Die Differenz von etwa \SI{33}{\kelvin} wird durch den natürlichen Treibhauseffekt der Atmosphäre verursacht. Lacis et al. (2010) schlussfolgern: \enquote{In round numbers, water vapor accounts for
about 50\% of Earth's greenhouse effect, with clouds contributing 25\%, CO2 20\%, and the minor GHGs and aerosols accounting for the remaining 5\%.}\cite{atmosphericCO2}, 
was zunächst vermuten lässt, dass CO\textsubscript{2} nicht die bedeutendste Rolle im Treibhauseffekt spielt. Jedoch bilden die nicht kondensierenden Treibhausgase 
CO\textsubscript{2}, O\textsubscript{3}, N\textsubscript{2}O und CH\textsubscript{4} die Grundlage des Treibhauseffekts, da sie im Gegensatz zu Wasserdampf nicht durch Niederschlag 
aus der Atmosphäre entfernt werden. Wasserdampf und Wolken (zusammen 75\%) wirken als schnelle Rückkopplungsmechanismen, 
deren Konzentration von der Temperatur und damit von den nicht kondensierenden Treibhausgasen abhängt\cite{atmosphericCO2}.

Die Anwendung des Wienschen Verschiebungsgesetzes (\autoref{eq:wiens_displacement_law}) auf Sonne ($T_{\text{Sonne}} = \SI{5772}{\kelvin}$\cite[S.6]{solarIrradiance}) 
und Erde ($T_{\text{Erde}} = \SI{288}{\kelvin}$\cite[S.11]{marshall2007atmosphere}) verdeutlicht die fundamentale spektrale Asymetrie zwischen solarer und terrestrischer Strahlung.
Während die solare Strahlung ihr Maximum bei $\lambda_{\text{Sonne,max}} = \SI{0.50}{\micro\meter}$ hat, ist das Maximum der terrestrischer Strahlung bei $\lambda_{\text{Erde,max}} = \SI{10.06}{\micro\meter}$
