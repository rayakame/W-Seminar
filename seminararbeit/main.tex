\documentclass[12pt,a4paper]{article}

% ===== PAKETE =====
\usepackage[ngerman]{babel}          % Deutsche Sprache
\usepackage[utf8]{inputenc}          % UTF-8 Kodierung
\usepackage[T1]{fontenc}             % Schriftkodierung
\usepackage{lmodern}                 % Schönere Schrift
\usepackage[left=2.5cm,right=2.5cm,top=2.5cm,bottom=2.5cm]{geometry}

% Layout & Formatierung
\usepackage{setspace}                % Zeilenabstand

\usepackage{parskip}                 % Absatzformatierung
\usepackage{fancyhdr}                % Kopf- und Fußzeilen

% Bilder & Grafiken
\usepackage{graphicx}                % Bilder einbinden
\usepackage{float}                   % Bessere Positionierung
\usepackage{caption}                 % Bildunterschriften
\usepackage{chngcntr}
\usepackage{mhchem}

\usepackage{tikz}
\usepackage{pgfplots}
\pgfplotsset{compat=1.18}
\usetikzlibrary{arrows.meta,decorations.markings,3d,positioning}

\usepackage{minted}
\setminted{
    breaklines=true,
    fontsize=\scriptsize,
    linenos=true,
    frame=single,
    %bgcolor=gray!10
}

% Tabellen
\usepackage{tabularx}                % Flexible Tabellen
\usepackage{booktabs}                % Schönere Tabellen

% PDF einbinden
\usepackage{pdfpages}                % Für Deckblatt

% Mathematik (falls benötigt)
\usepackage{amsmath}
\usepackage{amssymb}
\usepackage{siunitx}

\usepackage[autostyle=true]{csquotes} % quote \enquote


\numberwithin{equation}{section}
\numberwithin{figure}{section}
\numberwithin{table}{section}
\numberwithin{listing}{section}

\newenvironment{code}{\captionsetup{type=listing}}{}


% Links & Verweise
\usepackage{hyperref}                % Klickbare Links
\hypersetup{
    colorlinks=true,
    linkcolor=black,
    citecolor=black,
    urlcolor=blue,
    pdftitle={Strahlungsphysik des CO2-Treibhauseffekts},
    pdfauthor={Christopher Mehnert}
}

% Literaturverzeichnis
\usepackage[backend=biber,style=numeric,sorting=nty]{biblatex}
\addbibresource{literatur.bib}


% ===== EINSTELLUNGEN =====
\newboolean{abbildungen}
\newboolean{formeln}

\setboolean{abbildungen}{true}
\setboolean{formeln}{true}


%\onehalfspacing
\setstretch{1.15}
\setlength{\parindent}{0pt}


\pagestyle{fancy}
\fancyhf{}
\fancyhead[L]{\leftmark}
\fancyhead[R]{\thepage}
\renewcommand{\headrulewidth}{0.4pt}


\begin{document}

\pagenumbering{Roman}

\includepdf[pages={1,2,3}]{assets/deckblatt.pdf}


\pagenumbering{arabic}


\setcounter{page}{2}
\tableofcontents


\section*{Abkürzungsverzeichnis}
\begin{tabular}{@{}ll@{}}
    \textbf{Abkürzung} & \textbf{Bedeutung} \\
    IR & Infrarot \\
    UV & Ultraviolett \\
    \ce{CO2}  & Kohlenstoffdioxid \\
    \ce{H2O} & Wasser \\
    \ce{CH4} & Methan \\
    \ce{O2} & Sauerstoff \\
    \ce{N2} & Stickstoff \\
    \ce{O3} & Ozon \\
    \ce{N2O} & Distickstoffoxid \\
    $\si{ppm}$ & Parts per million (Teile pro Million) \\
\end{tabular}
\newpage

\section{Einleitung}
Der Klimawandel stellt eine der zentralen wissenschaftlichen und gesellschaftlichen Herausforderungen des 21. Jahrhunderts dar. 
Seit Beginn der industriellen Revolution ist die globale Mitteltemperatur um etwa $\SI{1.1}{\kelvin}$ angestiegen, wobei dieser Anstieg mit einer erhöhten 
atmosphärischen Kohlenstoffdioxid-Konzentration von präindustriellen 280~ppm auf mittlerweile über 420~ppm korreliert. 
Die physikalischen Grundlagen dieses Zusammenhangs wurden bereits im 19. Jahrhundert von Wissenschaftlern wie Joseph Fourier, John Tyndall 
und Svante Arrhenius erkannt, doch ein vollständiges Verständnis der Strahlungsprozesse erfordert ein detailliertes Studium der Wechselwirkungen 
zwischen Strahlung und atmosphärischen Spurengasen \cite{atmosphericCO2}.

Im Zentrum dieser Untersuchung steht die Frage: Welche strahlungsphysikalischen Mechanismen führen dazu, dass \ce{CO2} als Treibhausgas wirkt, 
und wie lässt sich dessen quantitativer Beitrag zum terrestrischen Energiehaushalt bestimmen? Diese Frage ist von fundamentaler Bedeutung für 
das Verständnis des Klimasystems, da \ce{CO2}, trotz seiner geringen atmosphärischen 
Konzentration, eine zentrale Rolle im Strahlungshaushalt der Erde spielt. Während Wasserdampf den größten Einzelbeitrag zum natürlichen 
Treibhauseffekt leistet, bilden die nicht kondensierenden Treibhausgase wie \ce{CO2} die stabilisierende Grundlage dieses Effekts, deren 
Bedeutung im Folgenden genauer analysiert wird.

Zur Beantwortung dieser Fragestellung werden drei zentrale Aspekte systematisch untersucht: die fundamentalen Strahlungsgesetze, 
die quantenmechanische Basis der wellenlängenselektiven Absorption sowie der Einfluss von \ce{CO2} auf die terrestrische Strahlungsbilanz 
mittels vereinfachter Treibhausmodelle.

Der methodische Ansatz basiert auf einer theoretischen Analyse der relevanten physikalischen Zusammenhänge. Die Berechnung des 
\ce{CO2}-Absorptionsspektrums erfolgt unter Verwendung spektroskopischer Daten aus der HITRAN-Datenbank \cite{hitran2020}, die mittels Python \cite{python} 
ausgewertet werden. Die Quantifizierung des Treibhauseffekts erfolgt durch analytische Strahlungsbilanzmodelle.

Strukturell ist diese Untersuchung wie folgt aufgebaut: Kapitel~2 behandelt die strahlungsphysikalischen Grundlagen und das Strahlungsgleichgewicht der Erde. 
Kapitel~3 untersucht die molekularen Mechanismen der Infrarotabsorption durch \ce{CO2}. Kapitel~4 quantifiziert den Einfluss von \ce{CO2} 
auf die globale Mitteltemperatur. Kapitel~5 fasst die Erkenntnisse zusammen und ordnet sie in den klimawissenschaftlichen Kontext ein.
\section{Physikalische Grundlagen der Wärmestrahlung}
Zum Verständnis des Treibhauseffekts ist es essenziell zu untersuchen, wie die Erde den Großteil ihrer Energie erhält und 
wie sie ihre überschüssige Energie wieder abgibt. Der Energiehaushalt der Erde wird hauptsächlich durch 
Strahlungsprozesse bestimmt: Die Erde erhält Energie durch Strahlung der Sonne und strahlt selbst wieder Energie in
den Weltraum \cite[S.~9--11]{marshall2007atmosphere}. Im folgenden Kapitel werden diese Strahlungsprozesse im Detail betrachtet, um die Grundlage
des Treibhauseffekts zu verstehen.

\subsection{Grundbegriffe der elektromagnetischen Strahlung}
\textit{Wärmestrahlung} bezeichnet die elektromagnetische Strahlung, die von Materie aufgrund ihrer thermischen
Energie emittiert wird. Elektromagnetische Strahlung ist eine Form der Energieübertragung, die sich im Raum ausbreitet und entweder 
als elektromagnetische Wellen (wie von der elektromagnetischen Wellentheorie beschrieben; \cite{electricityMagnetism})
oder als masselose Energiequanten, sogenannte \textit{Photonen} (wie von der Quantenmechanik beschrieben; \cite{quantumMechanics}),
aufgefasst werden kann. Keine der beiden Sichtweisen kann alle beobachteten Strahlungsphänomene vollständig beschreiben, weshalb beide
Konzepte komplementär verwendet werden. \cite[S.~1--2]{radiativeHeatTransfer}

Elektromagnetische Strahlung bewegt sich mit der \textit{Lichtgeschwindigkeit} $c = c_0/n$, wobei $c_0 = \SI{2.998e8}{\meter\per\second}$ 
die Lichtgeschwindigkeit im Vakuum ist und $n$ den Brechungsindex des Mediums bezeichnet. 
Da die in dieser Arbeit betrachteten Strahlungsprozesse hauptsächlich im Vakuum ($n \equiv 1$) oder 
in Luft ($n \approx 1.0002$) stattfinden, wird im Folgenden die Näherung $n = 1$ und somit $c = c_0$ verwendet \cite[S.~3]{radiativeHeatTransfer}.
Das elektromagnetische Spektrum wird nach der Wellenlänge in verschiedene Bereiche unterteilt, wobei für die Wärmeübertragung insbesondere die 
ultraviolette Strahlung (UV, $\lambda < \SI{380}{\nano\meter}$), das sichtbare Licht ($\SI{380}{\nano\meter} \leq \lambda \leq \SI{780}{\nano\meter}$) 
und die infrarote Strahlung (IR, $\lambda > \SI{780}{\nano\meter}$) von Bedeutung sind. Elektromagnetische Wellen werden durch folgende Eigenschaften charakterisiert:
\begin{table}[H]
    \centering
    \small
    \begin{tabularx}{1\textwidth}{l X}
        \toprule
        \textbf{Eigenschaft} & \textbf{Beschreibung und Einheit} \\
        \midrule
        Frequenz, $\nu$ & Anzahl der Schwingungen pro Zeiteinheit, $[\si{\hertz}] = [\si{\second}^{-1}]$ \\
        Wellenlänge, $\lambda$ & Abstand zwischen zwei Wellenmaxima, $[\si{\micro\meter}]$ oder $[\si{\nano\meter}]$ \\
        Wellenzahl, $\eta$ & Anzahl der Wellen pro Längeneinheit, $[\si{\centi\meter}^{-1}]$ \\
        \bottomrule
    \end{tabularx}
    \caption{Charakteristische Eigenschaften elektromagnetischer Wellen \cite[S.~4]{radiativeHeatTransfer}.}
    \label{tab:wave_properties}
\end{table}
\vspace{-0.5cm} 
\ifthenelse{\boolean{formeln}}{
  \begin{equation}
    \nu = \frac{c_0}{\lambda} = c_0\eta
    \quad \text{\cite[S.~4]{radiativeHeatTransfer}}
  \end{equation}
}{}
\ifthenelse{\boolean{abbildungen}}{
  \begin{figure}[H]
  \centering
  \begin{tikzpicture}[scale=0.75]
      \begin{scope}[xshift=0cm]
          \draw[->] (-0.5,0) -- (8,0) node[right] {$x$};
          \draw[->] (0,-1.5) -- (0,1.5) node[above] {$E$};
          \draw[thick, blue] plot[domain=0:7.42, samples=150] (\x, {sin(2*\x r)});
          
          \draw[<->, red, thick] (0,-1.8) -- (3.14,-1.8) node[midway, below] {$\lambda$};
          \draw[red, dashed] (3.14,0) -- (3.14,-1.8);
          
          \draw[<->, green!50!black, thick] (0.78,1.3) -- (7.06,1.3);
          \node[green!50!black, above] at (3.9,1.3) {1 cm};
          
          \draw[green!50!black, dashed] (0.78,0) -- (0.78,1.3);
          \draw[green!50!black, dashed] (7.06,0) -- (7.06,1.3);
          \node[green!50!black] at (5.57,-2.3) {$\eta = 2$ Wellen/cm = \SI{2}{\per\centi\meter}};
      \end{scope}
      
      \begin{scope}[xshift=9.5cm]
          \draw[->] (-0.5,0) -- (8,0) node[right] {$t$};
          \draw[->] (0,-1.5) -- (0,1.5) node[above] {$E$};
          \draw[thick, orange] plot[domain=0:7.42, samples=150] (\x, {sin(2*\x r)});
          
          \draw[<->, red, thick] (0,1.3) -- (6.28,1.3);
          \node[red, above] at (3.14,1.3) {1 s};
          \draw[red, dashed] (6.28,0) -- (6.28,1.3);

          \draw[<->, red, thick] (0,-1.8) -- (3.14,-1.8) node[midway, below] {$T$};
          \draw[red, dashed] (3.14,0) -- (3.14,-1.8);


          \node[red] at (5.57,-2.3) {$\nu = 2$ Wellen/s = \SI{2}{\hertz}};
      \end{scope}
  \end{tikzpicture}
  \caption{Visualisierung charakteristischer Eigenschaften elektromagnetischer Wellen. 
  Links: Räumliche Darstellung zeigt die Wellenlänge $\lambda$ und die Wellenzahl $\eta$. 
  Rechts: Zeitliche Darstellung zeigt die Periode $T$ und die Frequenz $\nu$.}
  \label{fig:wave_space_time}
  \end{figure}
}{}

\subsection{Strahlungsgesetze}
\label{sec:strahlungsgesetzte}
Jedes Medium emittiert elektromagnetische Strahlung isotrop in alle Richtungen. 
Die Intensität dieser Emission hängt sowohl von der Temperatur als auch von den Materialeigenschaften des Mediums ab. 
Der von einer Oberfläche abgegebene Strahlungswärmestrom wird als \textit{spezifische Ausstrahlung} $E$ bezeichnet und quantifiziert die Intensität der Emission.

Dabei wird zwischen der \textit{gesamten spezifischen Ausstrahlung} $E$ und der \textit{spektralen spezifischen Ausstrahlung} $E_{\nu}$ unterschieden:
\ifthenelse{\boolean{formeln}}{
  \begin{align*}
  E_{\nu} &\equiv \text{abgestrahlte Energie pro Zeit, Oberfläche und Frequenz (spektrale Intensität).} \\
  E &\equiv \text{abgestrahlte Energie pro Zeit und Oberfläche (Gesamtintensität).}
  \end{align*}
}{}
\cite[S.~6--7]{radiativeHeatTransfer}


\subsubsection{Das Plancksche Strahlungsgesetz}
Die spektrale spezifische Ausstrahlung eines ideal schwarzen Körpers $E_{b\nu}$ wird durch das Plancksche Strahlungsgesetz beschrieben. 
Es gibt an, wie viel Energie pro Zeit, Fläche und Frequenzintervall von einer ideal schwarzen Oberfläche bei einer bestimmten Temperatur $T$ emittiert wird. 
Dieses Gesetz wurde 1900 von Max Planck \cite{plancknormalspektrum} hergeleitet und ist heute als \textit{Plancksches Strahlungsgesetz} bekannt. 
Für eine schwarze Oberfläche ergibt sich die spektrale spezifische Ausstrahlung\cite[S.~7]{radiativeHeatTransfer} zu:
\ifthenelse{\boolean{formeln}}{
  \begin{equation}
    \label{eq:plancks_law_frequenzy}
    E_{b\nu}(T, \nu) = \frac{2\pi h \nu^3}{c_0^2} \cdot \frac{1}{e^{h\nu/kT}-1} 
  \end{equation}
}{}

Das Plancksche Strahlungsgesetz lässt sich auch in Abhängigkeit von der Wellenlänge $\lambda$ formulieren:
\ifthenelse{\boolean{formeln}}{
  \begin{equation}
    \label{eq:plancks_law_wavelength}
    E_{b\lambda}(T, \lambda) = \frac{2\pi h c_0^2}{\lambda^5} \cdot \frac{1}{e^{hc_0/\lambda kT}-1}
  \end{equation}
}{}
Dabei bezeichnet $h = \SI{6.626e-34}{\joule\second}$ das Plancksche Wirkungsquantum, 
$c_0 = \SI{2.998e8}{\meter\per\second}$ die Lichtgeschwindigkeit im Vakuum 
und $k = \SI{1.381e-23}{\joule\per\kelvin}$ die Boltzmann-Konstante \cite{codata2018}.

\subsubsection{Das Stefan-Boltzmann Gesetz}
\label{sec:stefan_boltzmann}
Die Integration der spektralen spezifischen Ausstrahlung über das gesamte 
elektromagnetische Spektrum liefert die \textit{Gesamtausstrahlung} $E$:
\ifthenelse{\boolean{formeln}}{
  \begin{equation}
    E = \int_{0}^{\infty}E_{\nu} \, d\nu
  \end{equation}
}{}
\cite{stefanBoltzmannLaw}

Für einen idealen schwarzen Körper wird $E_{b\nu}$ aus Gleichung 
\eqref{eq:plancks_law_frequenzy} in das Integral eingesetzt:
\ifthenelse{\boolean{formeln}}{
  \begin{equation*}
    E_b(T) = \int_{0}^{\infty} \frac{2\pi h \nu^3}{c_0^2} \cdot \frac{1}{e^{h\nu/kT}-1} \, d\nu
  \end{equation*}
}{}

Die Auswertung dieses Integrals erfordert komplexe Integrationstechniken und ist 
in Integraltabellen dokumentiert \cite[S.~13]{radiativeHeatTransfer}. 
Das Ergebnis ist das \textit{Stefan-Boltzmann Gesetz}:
\ifthenelse{\boolean{formeln}}{
  \begin{equation}
      \label{eq:stefan_boltzmann}
    E_b(T) = \frac{2\pi^5k^4}{15c_0^2h^3}T^4 = \sigma T^4 \quad \text{\cite{stefanBoltzmannLaw}}
  \end{equation}
}{}

Dabei bezeichnet $\sigma = \SI{5.670e-8}{\watt\per\meter\squared\per\kelvin\tothe{4}}$ 
die Stefan-Boltzmann Konstante \cite{codata2018}.

\subsubsection{Wiensches Verschiebungsgesetz}
\label{sec:wien}
Die Wellenlänge $\lambda_{max}$, bei der die spektrale spezifische Ausstrahlung eines idealen schwarzen Körpers $E_{b\lambda}$ 
mit der Temperatur $T$ ein Maximum erreicht, erhält man durch Ableitung der Gleichung \eqref{eq:plancks_law_wavelength} nach 
$\lambda$ und Nullsetzen dieser Ableitung (die vollständige Herleitung befindet sich in \autoref{sec:herleitung_wien}). 
Das daraus resultierende \textit{Wiensche Verschiebungsgesetz} besagt, dass sich das Maximum der spektralen Ausstrahlung mit steigender 
Temperatur zu kürzeren Wellenlängen verschiebt \cite[S.~101]{kraus2007atmosphäre}, wobei die Konstante 
$b = \SI{2.8978e-3}{\meter\kelvin}$ als \textit{Wiensche Verschiebungskonstante} bezeichnet wird \cite[S.~49]{codata2018}:
\ifthenelse{\boolean{formeln}}{
  \begin{equation}
    \label{eq:wiens_displacement_law}
    \lambda_{max} = \frac{b}{T}
  \end{equation}
}{}
\ifthenelse{\boolean{abbildungen}}{
  \begin{figure}[H]
      \centering
      \includegraphics[width=0.8\textwidth]{assets/wien_plot.pdf}
      \caption{Spektrale spezifische Ausstrahlung $E_{b\lambda}$ eines schwarzen Körpers nach dem Planckschen Strahlungsgesetz für verschiedene Temperaturen. Die gestrichelte Linie verbindet die Maxima der Planck-Kurven und verdeutlicht das Wiensche Verschiebungsgesetz. Grafik wurde mithilfe von \autocite{python} erstellt, der Code ist im Anhang \autoref{lst:wien} dokumentiert.}
      \label{fig:planck_wien}
  \end{figure}
}{}
\subsection{Strahlungsbilanz der Erde}
\label{sec:strahlungsbilanz_erde}
Die Erde bezieht nahezu ihre gesamte Energie von der Sonne. Um ein thermisches Gleichgewicht aufrechtzuerhalten, muss sie 
Energie mit derselben Rate wieder abstrahlen, mit der sie diese empfängt \cite[S.~9]{marshall2007atmosphere}:
\ifthenelse{\boolean{formeln}}{
  \begin{equation}
    P_{\text{in}} = P_{\text{out}}
    \label{eq:power_balance}
  \end{equation}
}{}
Die einfallende solare Bestrahlungsstärke am oberen Rand der Atmosphäre, die sogenannte \textit{Solarkonstante}, beträgt $S_0 = \SI{1361}{\watt\per\meter\squared}$\cite[S.~5--6]{solarIrradiance}.
Da die Querschnittsfläche der Erde, welche die solare Strahlung abfängt, $\pi r^2$ beträgt\cite[S.~11]{marshall2007atmosphere}, wobei $r = \SI{6.371e6}{\meter}$\cite[S.~56]{formelsammlung} der Erdradius ist, ergibt sich für die einfallende Strahlungsleistung:
\ifthenelse{\boolean{formeln}}{
  \begin{align}
    P_{\text{solar}} &= S_0\pi r^2 \nonumber\\
          &= \SI{1.735e17}{\watt}
    \label{eq:solar_power}
  \end{align}
}{}

Allerdings wird nicht die gesamte Strahlung von der Erde absorbiert, da ein Teil reflektiert wird. Der Anteil der reflektierten Strahlung wird als \textit{Albedo} $\alpha$ bezeichnet. 
Abbildung \ref{fig:albedo_map} zeigt, dass $\alpha$ von der reflektierenden Oberfläche abhängt (detaillierte Werte für verschiedene Oberflächen siehe Anhang \autoref{tab:albedos_surfaces}).
Im globalen Mittel wird ein Anteil von $\alpha_p \simeq 0{,}30$ der eingehenden Strahlung reflektiert. Diese Größe wird als \textit{planetare Albedo} bezeichnet \cite[S.~11]{marshall2007atmosphere}.
\ifthenelse{\boolean{abbildungen}}{
  \begin{figure}[H]
      \centering
      \includegraphics[width=0.8\textwidth]{assets/albedo_map.pdf}
      \caption{Räumliche Verteilung der Albedo an der Erdoberfläche \cite[S.~12]{marshall2007atmosphere}. Bild wurde mithilfe von \cite{bigjpg} hochskaliert.}
      \label{fig:albedo_map}
  \end{figure}
}{}

Daraus folgt für die von der Erde absorbierte Strahlungsleistung:
\ifthenelse{\boolean{formeln}}{
  \begin{align}
    P_{\text{in}} &= (1-\alpha_p)S_0\pi r^2 \nonumber\\
          &= \SI{1.215e17}{\watt}
    \label{eq:absorbed_power}
  \end{align}
}{}

Aufgrund des Energiegleichgewichts muss die von der Erde emittierte Strahlung die absorbierte Strahlung kompensieren. Dazu kann angenommen werden, dass sich die Erde
wie ein idealer schwarzer Körper mit gleichmäßiger Temperatur $T_e$ verhält (bekannt als \textit{effektive planetare Temperatur}) und dass das \textit{Stefan-Boltzmann-Gesetz} wie in \autoref{sec:stefan_boltzmann} anwendbar ist.

\autoref{eq:stefan_boltzmann} gibt die Gesamtausstrahlung $E_b(T)$ an, welche die Strahlungsintensität, also die ausgestrahlte Leistung pro Flächeneinheit, beschreibt: $E_b = P/A$ \cite{stefanBoltzmannLaw}. 
Die Erde strahlt über ihre gesamte Oberfläche $A = 4\pi r^2$ ab. Umgestellt nach der Strahlungsleistung ergibt sich \cite[S.~11]{marshall2007atmosphere}:\ifthenelse{\boolean{formeln}}{
  \begin{equation}
      P_{\text{out}} = \sigma \cdot 4\pi r^2 \cdot T_e^4
      \label{eq:earth_emission}
  \end{equation}
}{}

Durch Einsetzen von \autoref{eq:absorbed_power} und \autoref{eq:earth_emission} in \autoref{eq:power_balance} und Umstellen nach $T_e$ erhält man:
\ifthenelse{\boolean{formeln}}{
  \begin{equation}
      T_e = \left(\frac{S_0(1-\alpha_p)}{4\sigma}\right)^{1/4}
      \label{eq:effective_temperature}
  \end{equation}
}{}

Mit den Werten $S_0 = \SI{1361}{\watt\per\meter\squared}$, $\alpha_p = 0{,}30$ und $\sigma = \SI{5.670e-8}{\watt\per\meter\squared\per\kelvin\tothe{4}}$ ergibt sich:
\ifthenelse{\boolean{formeln}}{
  \begin{align}
      T_e &= \left(\frac{\SI{1361}{\watt\per\meter\squared} \cdot (1-0{,}30)}{4 \cdot \SI{5.670e-8}{\watt\per\meter\squared\per\kelvin\tothe{4}}}\right)^{1/4} \nonumber\\
      &= \SI{254.6}{\kelvin}
      \label{eq:effective_temperature_value}
  \end{align}
}{}

Diese Temperatur von etwa \SI{-18}{\celsius} liegt deutlich unter der gemessenen globalen Mitteltemperatur der Erdoberfläche von circa \SI{288}{\kelvin}\cite[S.~11]{marshall2007atmosphere}. 
Die Differenz von etwa \SI{33}{\kelvin} wird durch den natürlichen Treibhauseffekt der Atmosphäre verursacht. Lacis et al. (2010) schlussfolgern: \enquote{In round numbers, water vapor accounts for
about 50\% of Earth's greenhouse effect, with clouds contributing 25\%, CO2 20\%, and the minor GHGs and aerosols accounting for the remaining 5\%.}\cite{atmosphericCO2}, 
was zunächst vermuten lässt, dass CO\textsubscript{2} nicht die bedeutendste Rolle im Treibhauseffekt spielt. Dennoch bilden die nicht kondensierenden Treibhausgase 
\ce{CO2}, \ce{O3}, \ce{N20} und \ce{CH4} die Grundlage des Treibhauseffekts, da sie im Gegensatz zu Wasserdampf nicht durch Niederschlag 
aus der Atmosphäre entfernt werden. Wasserdampf und Wolken (zusammen 75\%) wirken als schnelle Rückkopplungsmechanismen, 
deren Konzentration von der Temperatur und damit von den nicht kondensierenden Treibhausgasen abhängt\cite{atmosphericCO2}.

Die Anwendung des Wienschen Verschiebungsgesetzes (\autoref{eq:wiens_displacement_law}) auf Sonne ($T_{\text{Sonne}} = \SI{5772}{\kelvin}$\cite[S.~6]{solarIrradiance}) 
und Erde ($T_{\text{Erde}} = \SI{288}{\kelvin}$\cite[S.~11]{marshall2007atmosphere}) verdeutlicht die fundamentale spektrale Asymmetrie zwischen solarer und terrestrischer Strahlung.
Während die solare Strahlung ihr Maximum bei $\lambda_{\text{Sonne,max}} = \SI{0.50}{\micro\meter}$ hat, liegt das Maximum der terrestrischen Strahlung bei $\lambda_{\text{Erde,max}} = \SI{10.06}{\micro\meter}$
(visualisiert in \autoref{fig:planck_earth_sun}).

\ifthenelse{\boolean{abbildungen}}{
  \begin{figure}[H]
      \centering
      \includegraphics[width=1\textwidth]{assets/planck_plot.pdf}
      \caption{Vergleich der spektralen spezifischen Ausstrahlung von Sonne und Erde nach dem Planckschen Strahlungsgesetz mit den Absorptionsbanden von CO\textsubscript{2}. Grafik erstellt mit \autocite{python}, Code dokumentiert in Anhang \autoref{lst:planck}.}
      \label{fig:planck_earth_sun}
  \end{figure}
}{}
Die in diesem Kapitel dargestellten Strahlungsgesetze bilden das Fundament für das Verständnis des Treibhauseffekts. 
Die Analyse der Strahlungsbilanz zeigt, dass die berechnete effektive planetare Temperatur deutlich 
unter der gemessenen globalen Mitteltemperatur liegt \cite[S.~11]{marshall2007atmosphere}. Diese Differenz 
resultiert aus dem natürlichen Treibhauseffekt, dessen zentraler Mechanismus die wellenlängenselektive 
Absorption durch Treibhausgase wie CO\textsubscript{2} ist, wie bereits in \autoref{fig:planck_earth_sun} ersichtlich ist.
Die molekularen Grundlagen der CO\textsubscript{2}-Absorption werden im folgenden Kapitel untersucht.
\section{Molekülphysik des CO\texorpdfstring{$_2$}{CO2}}
Die Wechselwirkung elektromagnetischer Strahlung mit atmosphärischen Molekülen 
ist ein quantenmechanisches Phänomen von erheblicher Komplexität. Im Rahmen dieser 
Arbeit liegt der Fokus auf den durch CO\textsubscript{2} bedingten Aspekten des Treibhauseffekts,
ohne die vollständige molekülphysikalische Theorie zu behandeln.

Für praktische Berechnungen atmosphärischer Strahlungstransfers stehen umfassende 
spektroskopische Datenbanken zur Verfügung. Die HITRAN-Datenbank \cite{hitran2020}\cite{DBhitran2020} enthält hochaufgelöste 
Absorptionslinien für über 50 atmosphärische Moleküle und bildet den Standard für 
quantitative Analysen. Die folgenden Abschnitte erläutern die physikalischen 
Grundlagen dieser Absorptionsprozesse, ohne die vollständige quantenmechanische 
Behandlung zu vertiefen.


\subsection{Grundlagen der molekularen Absorption}
Ein Photon, welches auf ein Gasmolekül trifft, kann entweder absorbiert oder gestreut werden. 
Die Streuung ändert die Ausbreitungsrichtung und gegebenenfalls die Energie  des Photons. Dieser Effekt ist für 
unsere Betrachtung jedoch von untergeordneter Bedeutung, weswegen wir uns auf die Absorption konzentrieren. Die 
Absorption eines Photons führt dazu, dass das Energieniveau des Moleküls angehoben wird. Umgekehrt kann ein Molekül 
sein Energieniveau senken, indem es ein Photon emittiert.

Die Gesamtenergie eines Moleküls setzt sich aus verschiedenen Beiträgen zusammen: der elektronischen Energie, 
die durch die Verteilung der Elektronen in den Molekülorbitalen bestimmt wird, der Rotationsenergie, die aus 
der Drehbewegung des gesamten Moleküls um seine Trägheitsachsen resultiert, sowie der Schwingungsenergie, 
die durch die periodischen Relativbewegungen der Atomkerne gegeneinander entsteht.

Die Quantenmechanik besagt, dass die Energieniveaus für molekulare Elektronenorbitale, ebenso wie die 
Energieniveaus für molekulare Rotation und Vibration, nur diskrete Werte annehmen können. Da die Energie 
eines Photons direkt proportional zu seiner Frequenz ist ($E = h\nu$), müssen Photonen eine bestimmte Frequenz 
haben, um absorbiert oder emittiert zu werden. Dies führt zu diskreten Spektrallinien.


\subsection{Molekülstruktur und Schwingungsmoden}
\label{sec:vibration_modes}
Wir haben bereits die drei möglichen Arten der Energieniveaus eines Moleküls kennengelernt. Jedoch ist die Energie,
welche nötig ist, um elektronische Übergänge zu bewirken, so groß, dass man elektromagnetische Strahlung mit sehr niedriger
Wellenlänge benötigt (zwischen $\SI{0.01}{\micro\meter}$ und $\SI{1.5}{\micro\meter}$)\cite[S.~287]{radiativeHeatTransfer}.
Da wir uns auf CO\textsubscript{2} konzentrieren wollen und dieses keine für uns relevanten Absorptionsbanden im Bereich unter 
$\SI{1.5}{\micro\meter}$ besitzt (siehe \autoref{fig:co2_spektrum_under_1_5}), werden wir diese Energieänderungen vernachlässigen.


\ifthenelse{\boolean{abbildungen}}{
    \begin{figure}[H]
        \centering
        \includegraphics[width=1\textwidth]{assets/co2_absorption_under_1_5.pdf}
        \caption{Absorptionsspektrum von CO\textsubscript{2} unter $\SI{1.5}{\micro\meter}$. Das einzige erkennbare Absorptionsband bei circa $\SI{1.435}{\micro\meter}$ ist nur knapp über $2\%$ Absorption. Daten kommen von der HITRAN Datenbank \autocite{DBhitran2020}, Grafik wurde mithilfe von \autocite{python} erstellt, der Code ist im Anhang \ref{lst:co2_spektrum_small} dokumentiert.}
        \label{fig:co2_spektrum_under_1_5}
    \end{figure}
}{}
Vibrationsübergänge erfordern deutlich weniger Energie, weshalb die zugehörigen Spektrallinien zwischen $\SI{1.5}{\micro\meter}$ und $\SI{10}{\micro\meter}$ auftreten.
Rotationsübergänge benötigen die geringste Energie und erscheinen daher bei Wellenlängen oberhalb von $\SI{10}{\micro\meter}$. In der Praxis werden 
Vibrationsübergänge jedoch fast immer von simultanen Rotationsübergängen begleitet. Daraus resultieren eng benachbarte Spektrallinien,
die teilweise überlappen und die sogenannten \textit{Vibrations-Rotations-Bänder} im Infrarotspektrum formen, auf welche wir uns konzentrieren werden.
Diese entstehen durch die \textit{Freiheitsgrade} eines Moleküls. Jedes Molekül bewegt sich im dreidimensionalen Raum, weswegen jedes Molekül drei 
translatorische Freiheitsgrade für die x-, y- und z-Richtung besitzt. Da jedes Atom eines Moleküls sich theoretisch im dreidimensionalen Raum bewegen kann, 
beträgt die Gesamtzahl der Freiheitsgrade eines Moleküls mit $N$ Atomen $3N$. Da die Atome im Molekül miteinander verbunden sind, können sie sich nicht 
unabhängig voneinander bewegen, sondern nur relativ zueinander. Diese Bewegungen entsprechen den Vibrations- und Rotations-Freiheitsgraden. Die Anzahl dieser 
inneren Freiheitsgrade beträgt $3N - 3$, also die Gesamtzahl der Freiheitsgrade minus die drei translatorischen Freiheitsgrade. Je nachdem, ob das Molekül linear 
oder nicht linear ist, werden diese inneren Freiheitsgrade unterschiedlich auf Rotations- und Vibrations-Freiheitsgrade aufgeteilt. Für lineare Moleküle gibt es 
zwei Rotations-Freiheitsgrade. Für CO\textsubscript{2} ergeben sich daher $3 \cdot 3 - 3 - 2 = 4$ Vibrations-Freiheitsgrade.

\ifthenelse{\boolean{abbildungen}}{
    \begin{figure}[H]
        \centering
        \includegraphics[width=1\textwidth]{assets/degrees_of_freedom.pdf}
        \caption{Rotations- und Vibrations-Freiheitsgrade für (a) zweiatomige, (b) linear dreiatomige und (c) nicht lineare dreiatomige Moleküle. \cite[S.~293]{radiativeHeatTransfer}}
        \label{fig:degrees_of_freedom}
    \end{figure}
}{}

\subsection{Quantenmechanische Grundlagen der Absorption}
\subsubsection{Rotationsübergänge}
Zur Vereinfachung wird angenommen, dass ein Molekül aus Punktmassen besteht, welche durch starre, masselose Stäbe verbunden sind, 
das sogenannte Modell des starren Rotators (rigid rotator model). Mit diesem Modell lassen sich 
die erlaubten Rotationsenergieniveaus mittels der Schrödinger-Gleichung berechnen. Die Lösung dieser Gleichung für lineare Moleküle besagt, 
dass die möglichen Energieniveaus wie folgt gegeben sind \cite[S.~293]{radiativeHeatTransfer}:
\ifthenelse{\boolean{formeln}}{
    \begin{equation}
        E_j = \frac{\hbar^2 }{2I}j(j + 1) = hc_0Bj(j+1), \quad j = 0,1,2,... \quad \text{\cite[S.~293]{radiativeHeatTransfer}}
        \label{eq:rigid_rotator}
    \end{equation}
}{}

Dabei ist $I$ das Trägheitsmoment des Moleküls und $\hbar$ die modifizierte Planck-Konstante. Zur Vereinfachung wurde
$B$ eingeführt, welches die \textit{Rotationskonstante} beschreibt. Die Rotationsquantenzahl $j$ beschreibt den Rotationszustand
des Moleküls. Für den Grundzustand gilt $j = 0$, bei dem das Molekül nicht rotiert. Je höher $j$ ist, desto schneller ist die 
Rotationsgeschwindigkeit des Moleküls \cite[S.~294]{radiativeHeatTransfer}.

Die für die Absorption bei einem Rotationsübergang erforderliche Energie lässt sich mit $\Delta E = E_{j + 1} - E_j$ berechnen, wodurch
auch die entsprechende Frequenz bestimmt werden kann ($E = h\nu$). Eine notwendige Bedingung für die Beobachtung solcher Übergänge ist jedoch, 
dass das Molekül ein permanentes elektrisches Dipolmoment besitzt. Da CO\textsubscript{2} aufgrund seiner symmetrischen linearen Struktur kein 
permanentes Dipolmoment aufweist, können reine Rotationsübergänge nicht beobachtet werden \cite[S.~294]{radiativeHeatTransfer}. Warum 
Rotationsübergänge trotzdem für CO\textsubscript{2} von Bedeutung sind, wird in \autoref{sec:combined_transitions} erläutert.

\subsubsection{Vibrationsübergänge}
Zur Vereinfachung wird angenommen, dass zwei Punktmassen durch eine perfekt elastische masselose Feder verbunden sind. Dieses Modell
wird der harmonische Oszillator genannt. Bei diesem Modell lassen sich die möglichen Energieniveaus ebenfalls mit einer Lösung der 
Schrödinger-Gleichung berechnen \cite[S.~294--259]{radiativeHeatTransfer}:

\ifthenelse{\boolean{formeln}}{
    \begin{equation}
        E_{\upsilon} = h\nu_e\left(\upsilon + \frac{1}{2}\right), \quad \upsilon = 0,1,2,... 
        \quad \text{\cite[S.~259]{radiativeHeatTransfer}}
        \label{eq:harmonic_oscillator}
    \end{equation}
}{}

Dabei ist $\nu_e$ die Eigenfrequenz der harmonischen Schwingung und $\upsilon$ ist die Vibrationsquantenzahl, welche
ähnlich wie die Rotationsquantenzahl den Vibrationszustand beschreibt. Dieses Modell lässt nur die Auswahlregel 
$\Delta \upsilon = \pm 1$ zu, sodass bei diesem Modell die Vibrationszustände nur um $1$ erhöht oder gesenkt werden können. 
Dies würde zu einer einzigen Spektrallinie bei der Frequenz der Eigenfrequenz führen \cite[S.~295]{radiativeHeatTransfer}:
\ifthenelse{\boolean{formeln}}{
    \begin{equation}
        \Delta E = E_{\upsilon + 1} - E_{\upsilon} = h\nu_e\left(\left(\upsilon + \frac{3}{2}\right) - \left(\upsilon + \frac{1}{2}\right)\right) = h\nu_e = \text{const}
    \end{equation}
}{}

\iffalse
$\Delta \upsilon = \pm 1$ folgt daraus das jede quantenmechanische Wellenfunktion der verschiedenen Vibrationszustände eine unterschiedliche Form hat.
Allgemein wechselt es zwischen symetrischen und asymetrischen Wellenformen, der Übergangsmoment (wie stark das Photon mit dem Molekül wechselwirkt)
wird durch ein Integral berechnet das beide Funktionen multipliziert.

Dieses Integral ist nur dann nicht null, wenn die Wellenfunktionen die richtige Symmetrie haben. Das passiert nur bei $\Delta \upsilon = \pm 1$
\fi

Leider ist das Modell eines harmonischen Oszillators deutlich ungenauer als das des starren Rotators. Dies lässt sich dadurch erklären,
dass bei einer perfekt elastischen Feder die Kraft linear mit der Auslenkung zunimmt, wohingegen die Kraft bei Atomen, die sich
relativ zueinander bewegen, nicht linear zunimmt. Wenn man daher eine komplexere Federkonstante in die Analyse einbezieht, führt dies zu
weiteren Termen in \autoref{eq:harmonic_oscillator}, wodurch sich die Auswahlregel zu $\Delta \upsilon = \pm 1, \pm 2, \pm 3, \ldots$ 
ändert. Dies resultiert in mehreren ungefähr gleichmäßig voneinander entfernten Spektrallinien. Der Übergang mit $\Delta \upsilon = \pm 1$ 
wird als \textit{Fundamentalübergang} bezeichnet und ist normalerweise bei weitem der stärkste, während die weiteren Übergänge mit 
$\Delta \upsilon = \pm 2$, $\Delta \upsilon = \pm 3$, $\ldots$ als Obertöne bezeichnet werden. Der Vibrationszustand eines Moleküls 
wird mithilfe der Vibrationsquantenzahlen beschrieben, bei CO\textsubscript{2} beispielsweise durch $(\upsilon_1, \upsilon_2, \upsilon_3)$. 
Obwohl CO\textsubscript{2} vier Vibrations-Freiheitsgrade besitzt (siehe \autoref{sec:vibration_modes}), 
werden nur drei Quantenzahlen benötigt. Der Grund hierfür ist, dass die zweite Vibrationsmode $\upsilon_2$ 
zweifach entartet ist: Es handelt sich um dieselbe Schwingung, die in zwei senkrecht zueinander 
stehenden Ebenen stattfindet \cite[S.~295--297]{radiativeHeatTransfer}.


\subsubsection{Kombinierte Übergänge}
\label{sec:combined_transitions}
Wären für CO\textsubscript{2} nur Vibrationsübergänge möglich, gäbe es im Absorptionsspektrum nur sehr schmale sichtbare Spektrallinien.
Dies ist jedoch nicht der Fall (siehe \autoref{fig:co2_spektrum}), da Vibrationsübergänge und Rotationsübergänge oft gleichzeitig stattfinden und zu den
oben genannten \textit{Vibrations-Rotations-Bändern} führen, welche aus vielen nah aneinanderliegenden Spektrallinien bestehen. Für Fundamentalübergänge 
liegt der Mittelpunkt dieser Linien wie beim Modell des harmonischen Oszillators bei der Eigenfrequenz des Vibrationsübergangs $\nu_e$. Dieser Punkt wird
als Bandmitte $\eta_0 = \nu_e/c_0$ bezeichnet \cite[S.~298]{radiativeHeatTransfer}.

Für das einfachste Modell eines starren Rotators (\autoref{eq:rigid_rotator}) kombiniert mit einem harmonischen Oszillator 
(\autoref{eq:harmonic_oscillator}) ergibt sich die Gesamtenergie eines Molekülzustands als Summe der Vibrationsenergie und 
der Rotationsenergie:

\ifthenelse{\boolean{formeln}}{
    \begin{equation}
        E_{\upsilon j} = h\nu_e\left(\upsilon + \frac{1}{2}\right) + hc_0B_{\upsilon }j(j+1), \quad \upsilon, j = 0,1,2,\ldots
        \label{eq:combined_harmonic_rigid}
    \end{equation}
}{}

Im Gegensatz zum idealisierten starren Rotator ist die Rotationskonstante in realen Molekülen 
vom Vibrationszustand abhängig, weshalb hier $B_{\upsilon}$ verwendet wird. Durch die gleichzeitige 
Änderung von Rotations- und Vibrationszustand entstehen bei einem Fundamentalübergang 
($\Delta \upsilon = \pm 1$) drei charakteristische Zweige: der P-Zweig ($\Delta j = -1$), 
der Q-Zweig ($\Delta j = 0$) und der R-Zweig ($\Delta j = +1$) \cite[S.~298]{radiativeHeatTransfer}. Die Wellenzahlen der 
Spektrallinien in diesen Zweigen sind gegeben durch:

\begin{subequations}
\begin{align}
    \eta_P(j) &= \eta_0 + j^2(B_{\upsilon+1} - B_{\upsilon}) - j(B_{\upsilon+1} + B_{\upsilon})\\
    \eta_Q(j) &= \eta_0 + j(j+1)(B_{\upsilon+1} - B_{\upsilon}) \\
    \eta_R(j) &= \eta_0 + j^2(B_{\upsilon+1} - B_{\upsilon}) + j(3B_{\upsilon+1} - B_{\upsilon}) + 2B_{\upsilon+1}
\end{align}
\end{subequations}

wobei $\eta_0$ die Bandmitte bezeichnet. Die detaillierte Herleitung dieser Formeln findet sich in \autoref{sec:herleitung_3_zweige}. 
Bei der Bandmitte $\eta_0$ selbst tritt keine Spektrallinie auf. Wenn $B_{\upsilon} = B_{\upsilon+1}$ dann gibt es keinen Q-Zweig,
dies ist oft der Fall bei linearen Molekülen wie CO\textsubscript{2}. In \autoref{fig:co2_spektrum_v3} sieht man die $P$- und $R$-Zweige des
$\upsilon_3$ Bandes von CO\textsubscript{2} visualisiert. Der $Q$-Zweig tritt nicht auf.

\ifthenelse{\boolean{abbildungen}}{
    \begin{figure}[H]
        \centering
        \includegraphics[width=1\textwidth]{assets/co2_absorption_v3_band.pdf}
        \caption{Absorptionsspektrum des $\upsilon_3$-Vibrationsbandes von CO\textsubscript{2}. Daten kommen von der HITRAN Datenbank \autocite{DBhitran2020}
         und von \cite[S.~39]{Shimanouchi1972}, Grafik wurde mithilfe von \autocite{python} erstellt, der Code ist im Anhang \ref{lst:co2_spektrum_v3} dokumentiert.}
        \label{fig:co2_spektrum_v3}
    \end{figure}
}{}



\subsection{Das CO\texorpdfstring{$_2$}{CO2}-Absorptionsspektrum}
\ifthenelse{\boolean{abbildungen}}{
    \begin{figure}[H]
        \centering
        \includegraphics[width=1\textwidth]{assets/co2_absorption.pdf}
        \caption{Absorptionsspektrum von CO\textsubscript{2}. Daten kommen von der HITRAN Datenbank \autocite{DBhitran2020}, Grafik wurde mithilfe von \autocite{python} erstellt, der Code ist im Anhang \ref{lst:co2_spektrum_small} dokumentiert.}
        \label{fig:co2_spektrum}
    \end{figure}
}{}
\section{Der Treibhauseffekt}
Die Analyse der Strahlungsbilanz in \autoref{sec:strahlungsbilanz_erde} zeigte,
dass die berechnete effektive planetare Temperatur deutlich unter der gemessenen 
globalen Mitteltemperatur liegt. Diese Differenz resultiert aus dem natürlichen
Treibhauseffekt. Der zentrale Mechanismus dieses Effekts und dessen Einfluss auf die Energiebilanz
der Erde werden im folgenden Kapitel untersucht.

\subsection{Mechanismus des Treibhauseffekts}
Der Treibhauseffekt entsteht durch die wellenlängenabhängige Absorption 
elektromagnetischer Strahlung in der Atmosphäre. Während die kurzwellige solare Strahlung 
die Atmosphäre weitgehend ungehindert durchdringt und die Erdoberfläche erwärmt, wird 
die von der Oberfläche emittierte langwellige Infrarotstrahlung teilweise von atmosphärischen 
Spurengasen absorbiert. Die Atmosphäre emittiert ihrerseits thermische Strahlung sowohl 
in Richtung Weltraum als auch zurück zur Erdoberfläche. Diese zusätzliche Gegenstrahlung 
führt zu einer Erhöhung der Oberflächentemperatur gegenüber dem Zustand ohne Atmosphäre.
 \cite{atmosphericCO2,marshall2007atmosphere}

Die physikalischen Grundlagen dieses Prozesses wurden bereits im 19. Jahrhundert erkannt: 
Joseph Fourier identifizierte 1824 den Mechanismus der atmosphärischen Wärmespeicherung, 
John Tyndall wies dies 1863 experimentell nach, und Svante Arrhenius quantifizierte 1896 erstmals den Zusammenhang 
zwischen \ce{CO2}-Konzentration und globaler Temperatur. \cite{atmosphericCO2}

\subsection{Treibhausmodelle}
Zur quantitativen Beschreibung des Treibhauseffekts dienen vereinfachte Strahlungsbilanzmodelle, 
die den Mechanismus schrittweise veranschaulichen und eine analytische Abschätzung der 
Oberflächentemperatur ermöglichen.

\subsubsection{Das einfache Treibhausmodell}
Das Modell setzt voraus, dass die Atmosphäre vollständig transparent für kurzwellige
Sonnenstrahlung ist, jedoch vollständig opak für Infrarotstrahlung (visualisiert 
in \autoref{fig:easy_greenhouse}). Folglich ist die Gesamtausstrahlung der Erde
von der Temperatur der Atmosphäre $T_a$ abhängig, da nicht die Erdoberfläche direkt, sondern
die Atmosphäre in den Weltraum strahlt. \cite[S.~14--15]{marshall2007atmosphere} Die Gesamtausstrahlung ergibt sich 
mithilfe des Stefan-Boltzmann-Gesetzes aus \autoref{eq:stefan_boltzmann}:
\ifthenelse{\boolean{formeln}}{
  \begin{equation}
      E_{\text{out}} = \sigma \cdot T_a^4 = A\uparrow 
  \end{equation}
}{}
\ifthenelse{\boolean{abbildungen}}{
    \begin{figure}[H]
        \centering
        \includegraphics[width=0.75\textwidth]{assets/athmosphäre_einfach.pdf}
        \caption{Schematische Darstellung des einfachen Treibhausmodells. \cite[S.~15]{marshall2007atmosphere} Bild wurde mithilfe von \cite{bigjpg} hochskaliert.}
        \label{fig:easy_greenhouse}
    \end{figure}
}{}
Die Energiezufuhr erfolgt durch die Sonneneinstrahlung. Da nur die Querschnittsfläche 
der Erde die Sonnenstrahlung abfängt, muss die Sonneneinstrahlung auf
die gesamte Erdoberfläche umgerechnet werden. \cite[S.~14]{marshall2007atmosphere} Der Faktor für den Anteil der reflektierten
Sonneneinstrahlung (planetare Albedo $\alpha_p$) bleibt gleich (siehe \autoref{sec:strahlungsbilanz_erde}).
\ifthenelse{\boolean{formeln}}{
  \begin{align}
      E_{\text{sun}} &= \frac{\text{interceptierte Sonnenstrahlung}}{\text{Oberfläche der Erde}} = \frac{S_0\pi r^2}{4\pi r^2} = \frac{S_0}{4}\\
      E_{\text{in}} &= (1-\alpha_p)\frac{S_0}{4}
  \end{align}
}{}
In diesem Treibhausmodell existieren zwei fundamentale Energiegleichgewichte:
$E_\text{in} = A\uparrow$, also Gesamteinstrahlung = Gesamtausstrahlung,
und $S\uparrow = E_\text{in} + A\downarrow$, also Ausstrahlung der 
Oberfläche = auf die Oberfläche einfallende Strahlung. \cite[S.~15]{marshall2007atmosphere}

Wird das Stefan-Boltzmann-Gesetz in das erste Energiegleichgewicht eingesetzt,
ergibt sich:
\ifthenelse{\boolean{formeln}}{
  \begin{equation}
      (1-\alpha_p)\frac{S_0}{4} = \sigma \cdot T_a^4
      \label{eq:simple_greenhouse_atm_balance}
  \end{equation}
}{}
Für das zweite Energiegleichgewicht folgt:
\ifthenelse{\boolean{formeln}}{
  \begin{equation}
      \sigma \cdot T_s^4 = (1-\alpha_p)\frac{S_0}{4} + \sigma \cdot T_a^4
      \label{eq:simple_greenhouse_surface_balance}
  \end{equation}
}{}
Durch Einsetzen von \autoref{eq:simple_greenhouse_atm_balance} in \autoref{eq:simple_greenhouse_surface_balance} ergibt sich:
\ifthenelse{\boolean{formeln}}{
  \begin{align*}
      \sigma \cdot T_s^4 &= 2\sigma \cdot T_a^4 \\
      T_s &= 2^{1/4}T_a
  \end{align*}
}{}

Dies zeigt, dass die Oberflächentemperatur in diesem Modell um den Faktor $2^{1/4} \approx 1.19$
größer ist als die effektive planetare Temperatur: $\SI{254.6}{\kelvin} \cdot 1{,}19 \approx \SI{303}{\kelvin}$.
Dieser Wert liegt näher an der globalen Mitteltemperatur der Erdoberfläche von $\SI{288}{\kelvin}$, 
stellt jedoch eine Überschätzung dar. Diese Diskrepanz resultiert aus der 
Modellannahme, dass die Atmosphäre sämtliche Infrarotstrahlung absorbiert. Wie in \autoref{sec:co2_spektrum} 
gezeigt wurde, absorbiert \ce{CO2} jedoch nicht den gesamten Infrarot-Wellenlängenbereich, 
sondern nur bestimmte spektrale Banden. Folglich muss das Treibhausmodell 
dahingehend modifiziert werden, dass es nur einen Teil der Infrarotstrahlung absorbiert.

\subsubsection{Das undichte Treibhausmodell}
Im erweiterten Modell wird angenommen, dass die Atmosphäre einen Teil der von der Oberfläche
emittierten Strahlung absorbiert und einen Teil transmittiert. Das Modell arbeitet mit
drei Temperaturen: $T_a$ bezeichnet die Temperatur der Atmosphäre,
$T_s$ die Oberflächentemperatur und $T_e$ die effektive planetare Temperatur 
(wie in \autoref{sec:strahlungsbilanz_erde} definiert), also die Temperatur der Erde 
als idealer schwarzer Körper ohne Atmosphäre. \cite[S.~16]{marshall2007atmosphere}
\ifthenelse{\boolean{abbildungen}}{
    \begin{figure}[H]
        \centering
        \includegraphics[width=0.75\textwidth]{assets/treibhausmodell_leaky.pdf}
        \caption{Schematische Darstellung des undichten Treibhausmodells. \cite[S.~17]{marshall2007atmosphere} Bild wurde mithilfe von \cite{bigjpg} hochskaliert.}
        \label{fig:leaky_greenhouse}
    \end{figure}
}{}
Das Energiegleichgewicht zwischen Gesamteinstrahlung und Gesamtausstrahlung
lautet in diesem Modell:
\ifthenelse{\boolean{formeln}}{
  \begin{equation}
      E_\text{in} = A\uparrow +  (1 - \varepsilon ) S\uparrow
  \end{equation}
}{}
wobei $\varepsilon$ den Absorptionsgrad der Atmosphäre im Infrarotbereich beschreibt. 
Das Energiegleichgewicht $S\uparrow = E_\text{in} + A\downarrow$ bleibt unverändert. \cite[S.~16]{marshall2007atmosphere}
Da $A\downarrow = A\uparrow$ gilt, folgt:
\ifthenelse{\boolean{formeln}}{
  \begin{equation*}
      S\uparrow = \frac{2}{2-\varepsilon}E_\text{in}
  \end{equation*}
}{}
Mit $S\uparrow = \sigma T_s^4$ und $E_\text{in} = \sigma T_e^4$ ergibt sich:
\ifthenelse{\boolean{formeln}}{
  \begin{equation}
      T_s = \left( \frac{2}{2-\varepsilon} \right)^{1/4}T_e
  \end{equation}
}{}
Die detaillierte Herleitung befindet sich in \autoref{sec:anhang_treibhausmodell}. Mit dieser
Gleichung lässt sich der reelle Absorptionsgrad der Atmosphäre bestimmen, da für
$T_s = \SI{288}{\kelvin}$ und $T_e = \SI{254.6}{\kelvin}$ 
Werte vorliegen. Aufgelöst nach $\varepsilon$ ergibt sich:
\ifthenelse{\boolean{formeln}}{
  \begin{equation}
    \varepsilon = 2 - \frac{2}{(T_s/T_e)^4} = \SI{77.8}{\percent}
  \end{equation}
}{}
\subsubsection{Berechnung des Absorptionsgrads von CO\texorpdfstring{$_2$}{2}}
Um den spezifischen Beitrag von \ce{CO2} zur Gesamtabsorption zu berechnen, lässt sich
die Absorption für jede Wellenlänge bestimmen und über den gesamten Wellenlängenbereich integrieren. \cite[S.~728]{bergman2020fundamentals}
\ifthenelse{\boolean{formeln}}{
  \begin{equation}
      \varepsilon_{\text{CO}_2}(T) = \frac{\int_0^\infty \varepsilon(\lambda) \cdot E_{b\lambda}(\lambda, T_s) \, d\lambda}{\int_0^\infty E_{b\lambda}(\lambda, T_s) \, d\lambda}
        = \frac{\int_0^\infty \varepsilon(\lambda) \cdot E_{b\lambda}(\lambda, T_s) \, d\lambda}{\sigma T_s^4}
    \end{equation}
}{}
Die wellenlängenabhängige Absorption von \ce{CO2} $\varepsilon(\lambda)$ wurde mithilfe der HITRAN-Datenbank bestimmt. \cite{DBhitran2020}
Die numerische Berechnung erfolgte mittels Python \cite{python}, der zugehörige Code ist in Anhang \ref{lst:co2_absorption} dargestellt. 
Daraus ergibt sich $\varepsilon_{\text{CO}_2} = \SI{23.01}{\percent}$. Nach dieser Berechnung trägt \ce{CO2} mit etwa $\SI{29.6}{\percent}$ zum
gesamten Absorptionsgrad bei. Experimentelle Daten zeigen jedoch, dass dieser Wert bei etwa $\SI{20}{\percent}$ liegt. \cite{atmosphericCO2} Dieser relative Fehler von ca. $\SI{48}{\percent}$
lässt sich durch verschiedene Faktoren erklären, die aus den Vereinfachungen des verwendeten Treibhausmodells resultieren. Die Abweichung resultiert 
hauptsächlich daraus, dass das Modell die Atmosphäre als homogene Schicht behandelt. Tatsächlich variieren Parameter wie Druck, Temperatur und Konzentration
stark mit der Höhe. Zudem stellt die isolierte Betrachtung von \ce{CO2} eine weitere Fehlerquelle dar, da sich das Absorptionsspektrum von \ce{CO2}
beispielsweise mit dem von H\textsubscript{2}O überschneidet.

\subsection{Spektrale Absorptionseigenschaften der Atmosphäre}
Es wurde bereits erkannt, dass die Annahme, die Atmosphäre absorbiere einen konstanten Anteil $\varepsilon$
der IR-Strahlung, eine starke Vereinfachung darstellt. In Wirklichkeit weist die Atmosphäre eine ausgeprägte spektrale Struktur auf:
\ifthenelse{\boolean{abbildungen}}{
    \begin{figure}[H]
        \centering
        \includegraphics[width=1\textwidth]{assets/atmosphäre_absorption.pdf}
        \caption{Absorptionsspektrum der Atmosphäre. \cite[S.~13]{marshall2007atmosphere} Bild wurde mithilfe von \cite{bigjpg} hochskaliert.}
        \label{fig:atmos_absorption}
    \end{figure}
}{}

Abbildung \ref{fig:atmos_absorption} zeigt, dass die Atmosphäre im sichtbaren Bereich (Maximum der 
Sonneneinstrahlung) nahezu transparent ist, während sie UV-Strahlung fast vollständig absorbiert. 
Im Infrarotbereich variiert die Absorption stark: Bei manchen Wellenlängen ist die Atmosphäre 
transparent (atmosphärisches Fenster), bei anderen nahezu undurchlässig. Bemerkenswert ist, dass 
die Hauptbestandteile N\textsubscript{2} und O\textsubscript{2} praktisch nicht zur Absorption beitragen, O\textsubscript{2} absorbiert 
nur im fernen UV und minimal im nahen IR, während N\textsubscript{2} über den gesamten relevanten Spektralbereich 
transparent bleibt. Der Treibhauseffekt wird somit ausschließlich durch Spurengase wie H\textsubscript{2}O, 
\ce{CO2} und \ce{CH4} verursacht. \cite[S.~14]{marshall2007atmosphere}

\subsection{Strahlungsantrieb durch CO\texorpdfstring{$_2$}{2}-Erhöhung}
Der Strahlungsantrieb $\Delta F$ wird definiert als die Änderung der Netto-Strahlungsenergieflussdichte
(in $\si{\watt\per\meter\squared}$) am Oberrand der Atmosphäre. Ein positiver Strahlungsantrieb
führt zu einer Energiezufuhr zum Klimasystem und damit zu einer Erwärmung, während ein negativer Strahlungsantrieb
eine Abkühlung bewirkt.

Für Änderungen der atmosphärischen \ce{CO2}-Konzentration ergibt sich der Strahlungsantrieb aus detaillierten
Strahlungstransfer-Berechnungen zu \cite[S.~2718]{newRadiativeForcingEstimates}:
\ifthenelse{\boolean{formeln}}{
  \begin{equation}
      \Delta F = 5{,}35 \cdot \ln\left(\frac{C}{C_0}\right) \quad [\si{\watt\per\meter\squared}]
      \label{eq:radiative_forcing_co2}
  \end{equation}
}{}
wobei $C$ die aktuelle \ce{CO2}-Konzentration und $C_0 = \SI{280}{ppm}$ die vorindustrielle Referenzkonzentration
bezeichnet. Die Temperaturantwort des Klimasystems auf einen Strahlungsantrieb wird durch die Klimasensitivität $\partial T_S / \partial Q$
beschrieben. \cite[S.~19]{marshall2007atmosphere}
\ifthenelse{\boolean{formeln}}{
  \begin{equation}
      \Delta T_s = \frac{\partial T_s}{\partial Q} \cdot \Delta F
      \label{eq:climate_sensitivity}
  \end{equation}
}{}

Für den einfachsten Fall ohne Rückkopplungen (nur Schwarzkörper-Rückstrahlung) ergibt 
sich aus der Ableitung des Stefan-Boltzmann-Gesetzes \cite[S.~19]{marshall2007atmosphere}:
\ifthenelse{\boolean{formeln}}{
  \begin{equation}
      \frac{\partial T_s}{\partial Q}\bigg|_{\text{BB}} = \left(4\sigma T_e^3\right)^{-1} \approx \SI{0.26}{\kelvin\per\watt\per\meter\squared}
      \quad \text{\cite[S.~19]{marshall2007atmosphere}}
  \end{equation}
}{}
Für $T_e = \SI{255}{\kelvin}$ folgt für eine \ce{CO2}-Verdopplung ($\Delta F \approx \SI{3.71}{\watt\per\meter\squared}$) 
eine direkte Temperaturerhöhung von nur etwa $\SI{0.99}{\kelvin}$.

Das reale Klimasystem weist jedoch starke Rückkopplungen auf, hauptsächlich durch Wasserdampf. Daher führt im realen Klimasystem eine Verdopplung
von \ce{CO2} zu einer Temperaturerhöhung von $\approx \SI{4}{\kelvin}$. \cite{atmosphericCO2}
\section{Fazit}
Die vorliegende Arbeit untersuchte die strahlungsphysikalischen Mechanismen des \ce{CO2}-Treibhauseffekts 
und dessen quantitativen Beitrag zum terrestrischen Energiehaushalt. Die eingangs gestellte Forschungsfrage 
kann auf Grundlage der theoretischen Analyse und numerischen Berechnungen wie folgt beantwortet werden:

Die Wirkung von \ce{CO2} als Treibhausgas beruht auf einem fundamentalen quantenmechanischen Prinzip: 
Das Molekül absorbiert elektromagnetische Strahlung ausschließlich bei jenen Wellenlängen, deren Photonenenergie 
exakt der Energiedifferenz zwischen erlaubten Vibrations-Rotations-Zuständen entspricht. Die Hauptabsorptionsbanden 
bei $\SI{4.26}{\micro\metre}$ ($\nu_3$-Band) und $\SI{15.0}{\micro\metre}$ ($\nu_2$-Band) liegen dabei im Spektralbereich 
der terrestrischen Wärmestrahlung, während \ce{CO2} für die kurzwellige Sonneneinstrahlung weitgehend transparent bleibt. 
Diese spektrale Selektivität erklärt den asymmetrischen Strahlungstransfer in der Atmosphäre.

Die quantitative Analyse mittels vereinfachter Strahlungsbilanzmodelle ergab, dass die Atmosphäre einen Absorptionsgrad 
von $\varepsilon = \SI{77.8}{\percent}$ aufweist. Der spezifische Beitrag von \ce{CO2} wurde durch Integration über das 
HITRAN-Absorptionsspektrum zu $\varepsilon_{\ce{CO2}} = \SI{23.01}{\percent}$ berechnet, was etwa $\SI{29.6}{\percent}$ 
des gesamten atmosphärischen Absorptionsgrads entspricht. Dieser Wert liegt über den experimentell ermittelten $\SI{20}{\percent}$, 
wobei die Diskrepanz hauptsächlich auf die Modellannahme einer homogenen Atmosphärenschicht zurückzuführen ist. In der 
Realität variieren Parameter wie Druck, Temperatur und Gaskonzentration stark mit der Höhe. Zudem vernachlässigt die 
isolierte Betrachtung von \ce{CO2} Überlappungseffekte mit anderen Treibhausgasen, insbesondere \ce{H2O}.

Trotz dieser Vereinfachungen demonstriert die Arbeit eindeutig, dass \ce{CO2} trotz seiner geringen atmosphärischen 
Konzentration von etwa $\SI{420}{ppm}$ einen messbaren und quantifizierbaren Beitrag zum Treibhauseffekt leistet. 
Die berechnete Temperaturdifferenz von $\SI{33}{\kelvin}$ zwischen der effektiven Planetentemperatur ($\SI{254.6}{\kelvin}$) 
und der gemessenen Oberflächentemperatur ($\SI{288}{\kelvin}$) verdeutlicht die fundamentale Bedeutung des 
atmosphärischen Treibhauseffekts für das Erdklima.

Die Ergebnisse bestätigen die bereits im 19. Jahrhundert von Fourier, Tyndall und Arrhenius erkannten physikalischen Grundlagen: 
\ce{CO2} fungiert als stabilisierendes, nicht kondensierendes Treibhausgas, dessen Konzentration die Basis für 
Rückkopplungsmechanismen wie Wasserdampf bildet.

Für zukünftige Untersuchungen wäre eine Erweiterung auf mehrschichtige Atmosphärenmodelle mit höhenabhängigen Parametern sowie 
die Berücksichtigung von Absorptionsüberlappungen zwischen verschiedenen Treibhausgasen wünschenswert. Dennoch liefert die 
vorliegende Analyse ein fundiertes physikalisches Verständnis der strahlungsphysikalischen Prozesse, die den Zusammenhang 
zwischen atmosphärischer \ce{CO2}-Konzentration und globaler Temperatur bestimmen.
\newpage
\section{Anhang}

\subsection{Literaturverzeichnis}
\printbibliography[heading=none]

\newpage
\pagenumbering{Alph}
\subsection{Daten und Tabellen}

\begin{table}[H]
    \centering
    \begin{tabularx}{0.8\textwidth}{l X}
        \toprule
        \textbf{Art der Oberfläche} & \textbf{Albedo (\%)} \\
        \midrule
        Ozean & 2-10 \\
        Wald & 6-18 \\
        Städte & 14-18 \\
        Grass & 7-25 \\
        Acker & 10-20 \\
        Natürliche Graslandökosysteme & 16-20 \\
        Wüste(Sand) & 35-45 \\
        Eis & 20-70 \\
        Wolken (dünn) & 30 \\
        Wolken (dick) & 60-70 \\
        Schnee (alt) & 40-60 \\
        Schnee (frisch) & 75-95 \\
        \bottomrule
    \end{tabularx}
    \caption{Albedos für Unterschiedliche Oberflächen\cite[S.11]{marshall2007atmosphere}.}
    \label{tab:albedos_surfaces}
\end{table}

\subsection{Formeln und Herleitungen}
\subsubsection{Herleitung der 3 Übergangsformeln}
\label{sec:herleitung_3_zweige}
Die in \autoref{sec:combined_transitions} beschriebenen 3 Gleichungen für die verschiedenen Zweige eines Vibrations-Rotations-Bands kommen wie folgt zustande:

Als Grundlage wird \autoref{eq:combined_harmonic_rigid} genommen und da wir ja einen Übergang der Energieniveaus brauchen, 
gehen wir erstmal von einem anfänglichen $\upsilon''$, $j''$ und finalen $\upsilon'$, $j'$ Zustand aus:
\ifthenelse{\boolean{formeln}}{
    \begin{align*}
        \Delta E &= E_{j'\upsilon'} - E_{j''\upsilon''} \\
                 &= h\nu_e\left(\upsilon' + \frac{1}{2}\right) + hc_0B_{\upsilon'}j'(j'+1) \\
                 &\quad - \left[h\nu_e\left(\upsilon'' + \frac{1}{2}\right) + hc_0B_{\upsilon''}j''(j''+1)\right] \\
                 &= h\nu_e(\upsilon' - \upsilon'') + hc_0\left[B_{\upsilon'}j'(j'+1) - B_{\upsilon''}j''(j''+1)\right]
    \end{align*}
}{}

Wir betrachten hier nur den Fundamentalübergang einer Absorption, deswegen gilt $\Delta \upsilon = +1$ was zu $\upsilon' = \upsilon'' + 1$ führt. 
Die Notation für beide Zustände wird jetzt zu $\upsilon'' = \upsilon$ und $\upsilon' = \upsilon + 1$ vereinfacht. In die Formel eingesetzt:
\ifthenelse{\boolean{formeln}}{
    \begin{align*}
        \Delta E &= h\nu_e(\upsilon + 1 - \upsilon) + hc_0\left[B_{\upsilon+1}j'(j'+1) - B_{\upsilon}j''(j''+1)\right] \\
                    &= h\nu_e + hc_0\left[B_{\upsilon+1}j'(j'+1) - B_{\upsilon}j''(j''+1)\right]
    \end{align*}
}{}

Umgerechnet in Wellenzahl ergibt sich:
\ifthenelse{\boolean{formeln}}{
    \begin{equation*}
        \Delta\eta = \frac{\Delta E}{hc_0} = \frac{\nu_e}{c_0} + B_{\upsilon+1}j'(j'+1) - B_{\upsilon}j''(j''+1)
    \end{equation*}
}{}

Da $\eta_0 = \nu_e/c_0$:
\ifthenelse{\boolean{formeln}}{
    \begin{equation}
        \Delta\eta = \eta_0 + B_{\upsilon+1}j'(j'+1) - B_{\upsilon}j''(j''+1)
        \label{eq:delta_eta}
    \end{equation}
}{}

Ähnlich wie bei einem vorherigen Schritt führen wir wieder die Vereinfachung $j'' = j$ ein, $j'$ kann auch wieder vereinfacht werden,
je nach Zweig zu $j' = j + 1$, $j' = j$ oder $j' = j - 1$. Für den $P$ Zweig ($j' = j - 1$) ergibt das:
\ifthenelse{\boolean{formeln}}{
        \begin{align*}
j'(j'+1) &= (j-1)(j-1+1) \\
            &= (j-1) \cdot j \\
            &= j^2 - j 
    \end{align*}
}{}

Eingesetzt in \autoref{eq:delta_eta}: 

\ifthenelse{\boolean{formeln}}{
    \begin{align*}
        \eta_P &= \eta_0 + B_{\upsilon+1}(j^2 - j) - B_{\upsilon}j(j+1) \\
                  &= \eta_0 + B_{\upsilon+1}j^2 - B_{\upsilon+1}j - B_{\upsilon}j^2 - B_{\upsilon}j \\
                  &= \eta_0 + j^2(B_{\upsilon+1} - B_{\upsilon}) - j(B_{\upsilon+1} + B_{\upsilon})
    \end{align*}
}{}


\subsection{Quellcode}


\renewcommand{\listoflistingscaption}{{\large Liste aller Quellcodes}}
\listoflistings

\newpage

\begin{code}
\inputminted{python}{../simulationen/formeln/formeln.py}
\caption[Verschiedene implementierungen für physikalische Gesetzte.]{Verschiedene implementierungen für physikalische Gesetzte welche von anderen Quellcodes benutzt werden.}
\label{lst:formeln}
\end{code}

\newpage

\begin{code}
\inputminted{python}{../simulationen/utils.py}
\caption[Verschiedene Hilfs-Funktionen]{Verschiedene Funktionen die in mehreren anderen Quellcodes benötigt werden.}
\label{lst:utils}
\end{code}

\newpage

\begin{code}
\inputminted{python}{../simulationen/wien/__main__.py}
\caption[Planck Funktion für verschiedene Temperaturen mit Wienschem Verschiebungsgesetz.]{Planck Funktion für verschiedene Temperaturen mit Wienschem Verschiebungsgesetz. Benutzt Formeln aus \autoref{lst:formeln}.}
\label{lst:wien}
\end{code}

\newpage

\begin{code}
\inputminted{python}{../simulationen/planck/__main__.py}
\caption[Planck Funktion für Sonne \& Erde mit CO2 Absorptionsbänden.]{Planck Funktion für Sonne \& Erde mit CO2 Absorptionsbänden. Benutzt Formeln aus \autoref{lst:formeln}.}
\label{lst:planck}
\end{code}

\newpage

\begin{code}
\inputminted{python}{../simulationen/co2_spektrum/__main__.py}
\caption[Visualisierung des CO\texorpdfstring{$_2$}{2}-Absorptionsspektrums]{Visualisierung des CO\texorpdfstring{$_2$}{2}-Absorptionsspektrums mithilfe von HITRAN Daten \autocite{DBhitran2020}. Benutzt Funktionen aus \autoref{lst:utils}.}
\label{lst:co2_spektrum}
\end{code}

\newpage

\begin{code}
\inputminted{python}{../simulationen/co2_spektrum_under_1_5/__main__.py}
\caption[Visualisierung des CO\texorpdfstring{$_2$}{2}-Absorptionsspektrums unter 1.5 micrometern]{Visualisierung des CO\texorpdfstring{$_2$}{2}-Absorptionsspektrums unterhalb von $\SI{1.5}{\micro\meter}$ mithilfe von HITRAN Daten \autocite{DBhitran2020}. Benutzt Funktionen aus \autoref{lst:utils}.}
\label{lst:co2_spektrum_small}
\end{code}





\end{document}
